%% bare_conf.tex
%% V1.4b
%% 2015/08/26
%% by Michael Shell
%% See:
%% http://www.michaelshell.org/
%% for current contact information.
%%
%% This is a skeleton file demonstrating the use of IEEEtran.cls
%% (requires IEEEtran.cls version 1.8b or later) with an IEEE
%% conference paper.
%%
%% Support sites:
%% http://www.michaelshell.org/tex/ieeetran/
%% http://www.ctan.org/pkg/ieeetran
%% and
%% http://www.ieee.org/

%%*************************************************************************
%% Legal Notice:
%% This code is offered as-is without any warranty either expressed or
%% implied; without even the implied warranty of MERCHANTABILITY or
%% FITNESS FOR A PARTICULAR PURPOSE! 
%% User assumes all risk.
%% In no event shall the IEEE or any contributor to this code be liable for
%% any damages or losses, including, but not limited to, incidental,
%% consequential, or any other damages, resulting from the use or misuse
%% of any information contained here.
%%
%% All comments are the opinions of their respective authors and are not
%% necessarily endorsed by the IEEE.
%%
%% This work is distributed under the LaTeX Project Public License (LPPL)
%% ( http://www.latex-project.org/ ) version 1.3, and may be freely used,
%% distributed and modified. A copy of the LPPL, version 1.3, is included
%% in the base LaTeX documentation of all distributions of LaTeX released
%% 2003/12/01 or later.
%% Retain all contribution notices and credits.
%% ** Modified files should be clearly indicated as such, including  **
%% ** renaming them and changing author support contact information. **
%%*************************************************************************


% *** Authors should verify (and, if needed, correct) their LaTeX system  ***
% *** with the testflow diagnostic prior to trusting their LaTeX platform ***
% *** with production work. The IEEE's font choices and paper sizes can   ***
% *** trigger bugs that do not appear when using other class files.       ***                          ***
% The testflow support page is at:
% http://www.michaelshell.org/tex/testflow/



\documentclass[conference]{IEEEtran}
\usepackage{cite}
\usepackage{graphicx}
\usepackage{hyperref}
\usepackage{float}

% Some Computer Society conferences also require the compsoc mode option,
% but others use the standard conference format.
%
% If IEEEtran.cls has not been installed into the LaTeX system files,
% manually specify the path to it like:
% \documentclass[conference]{../sty/IEEEtran}





% Some very useful LaTeX packages include:
% (uncomment the ones you want to load)


% *** MISC UTILITY PACKAGES ***
%
%\usepackage{ifpdf}
% Heiko Oberdiek's ifpdf.sty is very useful if you need conditional
% compilation based on whether the output is pdf or dvi.
% usage:
% \ifpdf
%   % pdf code
% \else
%   % dvi code
% \fi
% The latest version of ifpdf.sty can be obtained from:
% http://www.ctan.org/pkg/ifpdf
% Also, note that IEEEtran.cls V1.7 and later provides a builtin
% \ifCLASSINFOpdf conditional that works the same way.
% When switching from latex to pdflatex and vice-versa, the compiler may
% have to be run twice to clear warning/error messages.






% *** CITATION PACKAGES ***
%
%\usepackage{cite}
% cite.sty was written by Donald Arseneau
% V1.6 and later of IEEEtran pre-defines the format of the cite.sty package
% \cite{} output to follow that of the IEEE. Loading the cite package will
% result in citation numbers being automatically sorted and properly
% "compressed/ranged". e.g., [1], [9], [2], [7], [5], [6] without using
% cite.sty will become [1], [2], [5]--[7], [9] using cite.sty. cite.sty's
% \cite will automatically add leading space, if needed. Use cite.sty's
% noadjust option (cite.sty V3.8 and later) if you want to turn this off
% such as if a citation ever needs to be enclosed in parenthesis.
% cite.sty is already installed on most LaTeX systems. Be sure and use
% version 5.0 (2009-03-20) and later if using hyperref.sty.
% The latest version can be obtained at:
% http://www.ctan.org/pkg/cite
% The documentation is contained in the cite.sty file itself.






% *** GRAPHICS RELATED PACKAGES ***
%
\ifCLASSINFOpdf
  % \usepackage[pdftex]{graphicx}
  % declare the path(s) where your graphic files are
  % \graphicspath{{../pdf/}{../jpeg/}}
  % and their extensions so you won't have to specify these with
  % every instance of \includegraphics
  % \DeclareGraphicsExtensions{.pdf,.jpeg,.png}
\else
  % or other class option (dvipsone, dvipdf, if not using dvips). graphicx
  % will default to the driver specified in the system graphics.cfg if no
  % driver is specified.
  % \usepackage[dvips]{graphicx}
  % declare the path(s) where your graphic files are
  % \graphicspath{{../eps/}}
  % and their extensions so you won't have to specify these with
  % every instance of \includegraphics
  % \DeclareGraphicsExtensions{.eps}
\fi
% graphicx was written by David Carlisle and Sebastian Rahtz. It is
% required if you want graphics, photos, etc. graphicx.sty is already
% installed on most LaTeX systems. The latest version and documentation
% can be obtained at: 
% http://www.ctan.org/pkg/graphicx
% Another good source of documentation is "Using Imported Graphics in
% LaTeX2e" by Keith Reckdahl which can be found at:
% http://www.ctan.org/pkg/epslatex
%
% latex, and pdflatex in dvi mode, support graphics in encapsulated
% postscript (.eps) format. pdflatex in pdf mode supports graphics
% in .pdf, .jpeg, .png and .mps (metapost) formats. Users should ensure
% that all non-photo figures use a vector format (.eps, .pdf, .mps) and
% not a bitmapped formats (.jpeg, .png). The IEEE frowns on bitmapped formats
% which can result in "jaggedy"/blurry rendering of lines and letters as
% well as large increases in file sizes.
%
% You can find documentation about the pdfTeX application at:
% http://www.tug.org/applications/pdftex





% *** MATH PACKAGES ***
%
%\usepackage{amsmath}
% A popular package from the American Mathematical Society that provides
% many useful and powerful commands for dealing with mathematics.
%
% Note that the amsmath package sets \interdisplaylinepenalty to 10000
% thus preventing page breaks from occurring within multiline equations. Use:
%\interdisplaylinepenalty=2500
% after loading amsmath to restore such page breaks as IEEEtran.cls normally
% does. amsmath.sty is already installed on most LaTeX systems. The latest
% version and documentation can be obtained at:
% http://www.ctan.org/pkg/amsmath





% *** SPECIALIZED LIST PACKAGES ***
%
%\usepackage{algorithmic}
% algorithmic.sty was written by Peter Williams and Rogerio Brito.
% This package provides an algorithmic environment fo describing algorithms.
% You can use the algorithmic environment in-text or within a figure
% environment to provide for a floating algorithm. Do NOT use the algorithm
% floating environment provided by algorithm.sty (by the same authors) or
% algorithm2e.sty (by Christophe Fiorio) as the IEEE does not use dedicated
% algorithm float types and packages that provide these will not provide
% correct IEEE style captions. The latest version and documentation of
% algorithmic.sty can be obtained at:
% http://www.ctan.org/pkg/algorithms
% Also of interest may be the (relatively newer and more customizable)
% algorithmicx.sty package by Szasz Janos:
% http://www.ctan.org/pkg/algorithmicx




% *** ALIGNMENT PACKAGES ***
%
%\usepackage{array}
% Frank Mittelbach's and David Carlisle's array.sty patches and improves
% the standard LaTeX2e array and tabular environments to provide better
% appearance and additional user controls. As the default LaTeX2e table
% generation code is lacking to the point of almost being broken with
% respect to the quality of the end results, all users are strongly
% advised to use an enhanced (at the very least that provided by array.sty)
% set of table tools. array.sty is already installed on most systems. The
% latest version and documentation can be obtained at:
% http://www.ctan.org/pkg/array


% IEEEtran contains the IEEEeqnarray family of commands that can be used to
% generate multiline equations as well as matrices, tables, etc., of high
% quality.




% *** SUBFIGURE PACKAGES ***
%\ifCLASSOPTIONcompsoc
%  \usepackage[caption=false,font=normalsize,labelfont=sf,textfont=sf]{subfig}
%\else
%  \usepackage[caption=false,font=footnotesize]{subfig}
%\fi
% subfig.sty, written by Steven Douglas Cochran, is the modern replacement
% for subfigure.sty, the latter of which is no longer maintained and is
% incompatible with some LaTeX packages including fixltx2e. However,
% subfig.sty requires and automatically loads Axel Sommerfeldt's caption.sty
% which will override IEEEtran.cls' handling of captions and this will result
% in non-IEEE style figure/table captions. To prevent this problem, be sure
% and invoke subfig.sty's "caption=false" package option (available since
% subfig.sty version 1.3, 2005/06/28) as this is will preserve IEEEtran.cls
% handling of captions.
% Note that the Computer Society format requires a larger sans serif font
% than the serif footnote size font used in traditional IEEE formatting
% and thus the need to invoke different subfig.sty package options depending
% on whether compsoc mode has been enabled.
%
% The latest version and documentation of subfig.sty can be obtained at:
% http://www.ctan.org/pkg/subfig




% *** FLOAT PACKAGES ***
%
%\usepackage{fixltx2e}
% fixltx2e, the successor to the earlier fix2col.sty, was written by
% Frank Mittelbach and David Carlisle. This package corrects a few problems
% in the LaTeX2e kernel, the most notable of which is that in current
% LaTeX2e releases, the ordering of single and double column floats is not
% guaranteed to be preserved. Thus, an unpatched LaTeX2e can allow a
% single column figure to be placed prior to an earlier double column
% figure.
% Be aware that LaTeX2e kernels dated 2015 and later have fixltx2e.sty's
% corrections already built into the system in which case a warning will
% be issued if an attempt is made to load fixltx2e.sty as it is no longer
% needed.
% The latest version and documentation can be found at:
% http://www.ctan.org/pkg/fixltx2e


%\usepackage{stfloats}
% stfloats.sty was written by Sigitas Tolusis. This package gives LaTeX2e
% the ability to do double column floats at the bottom of the page as well
% as the top. (e.g., "\begin{figure*}[!b]" is not normally possible in
% LaTeX2e). It also provides a command:
%\fnbelowfloat
% to enable the placement of footnotes below bottom floats (the standard
% LaTeX2e kernel puts them above bottom floats). This is an invasive package
% which rewrites many portions of the LaTeX2e float routines. It may not work
% with other packages that modify the LaTeX2e float routines. The latest
% version and documentation can be obtained at:
% http://www.ctan.org/pkg/stfloats
% Do not use the stfloats baselinefloat ability as the IEEE does not allow
% \baselineskip to stretch. Authors submitting work to the IEEE should note
% that the IEEE rarely uses double column equations and that authors should try
% to avoid such use. Do not be tempted to use the cuted.sty or midfloat.sty
% packages (also by Sigitas Tolusis) as the IEEE does not format its papers in
% such ways.
% Do not attempt to use stfloats with fixltx2e as they are incompatible.
% Instead, use Morten Hogholm'a dblfloatfix which combines the features
% of both fixltx2e and stfloats:
%
% \usepackage{dblfloatfix}
% The latest version can be found at:
% http://www.ctan.org/pkg/dblfloatfix




% *** PDF, URL AND HYPERLINK PACKAGES ***
%
%\usepackage{url}
% url.sty was written by Donald Arseneau. It provides better support for
% handling and breaking URLs. url.sty is already installed on most LaTeX
% systems. The latest version and documentation can be obtained at:
% http://www.ctan.org/pkg/url
% Basically, \url{my_url_here}.




% *** Do not adjust lengths that control margins, column widths, etc. ***
% *** Do not use packages that alter fonts (such as pslatex).         ***
% There should be no need to do such things with IEEEtran.cls V1.6 and later.
% (Unless specifically asked to do so by the journal or conference you plan
% to submit to, of course. )


% correct bad hyphenation here
\hyphenation{op-tical net-works semi-conduc-tor}
\usepackage{listings}
\usepackage{xcolor}
\usepackage{algorithm}
\usepackage{algpseudocode}
\usepackage{amsmath}

\begin{document}
%
% paper title
% Titles are generally capitalized except for words such as a, an, and, as,
% at, but, by, for, in, nor, of, on, or, the, to and up, which are usually
% not capitalized unless they are the first or last word of the title.
% Linebreaks \\ can be used within to get better formatting as desired.
% Do not put math or special symbols in the title.
\title{SIAER(Simulador de Impactos de Asteroides y Evaluación de Riesgos): Una página web 
para calcular las consecuencias medioambientales regionales del impacto de un asteroide en la Tierra}


% author names and affiliations
% use a multiple column layout for up to three different
% affiliations
\author{
    \IEEEauthorblockN{Bendezu Pastrana, Tommy}
    \IEEEauthorblockA{Dept. de Ing. Mecatrónica\\
    Universidad Continental\\
    Huancayo, Perú\\
    tommyhanss007@gmail.com}

\and

    \IEEEauthorblockN{Meza Millan, Martin}
    \IEEEauthorblockA{Dept. de Ing. Civil\\
    Universidad Continental\\
    Huancayo, Perú\\
    mezamilland@gmail.com}

\and

    \IEEEauthorblockN{Ildefonso Santos, Steve}
    \IEEEauthorblockA{Dept. de Computer Science\\
    UTEC\\
    Huancayo, Perú\\
    steve.ildefonso@utec.edu.pe}

\and

    \IEEEauthorblockN{CUALQUIERA}
    \IEEEauthorblockA{Dept. de \\
    NS\\
    , Perú\\
    CORREO}
}

% conference papers do not typically use \thanks and this command
% is locked out in conference mode. If really needed, such as for
% the acknowledgment of grants, issue a \IEEEoverridecommandlockouts
% after \documentclass

% for over three affiliations, or if they all won't fit within the width
% of the page, use this alternative format:
% 
%\author{\IEEEauthorblockN{Michael Shell\IEEEauthorrefmark{1},
%Homer Simpson\IEEEauthorrefmark{2},
%James Kirk\IEEEauthorrefmark{3}, 
%Montgomery Scott\IEEEauthorrefmark{3} and
%Eldon Tyrell\IEEEauthorrefmark{4}}
%\IEEEauthorblockA{\IEEEauthorrefmark{1}School of Electrical and Computer Engineering\\
%Georgia Institute of Technology,
%Atlanta, Georgia 30332--0250\\ Email: see http://www.michaelshell.org/contact.html}
%\IEEEauthorblockA{\IEEEauthorrefmark{2}Twentieth Century Fox, Springfield, USA\\
%Email: homer@thesimpsons.com}
%\IEEEauthorblockA{\IEEEauthorrefmark{3}Starfleet Academy, San Francisco, California 96678-2391\\
%Telephone: (800) 555--1212, Fax: (888) 555--1212}
%\IEEEauthorblockA{\IEEEauthorrefmark{4}Tyrell Inc., 123 Replicant Street, Los Angeles, California 90210--4321}}




% use for special paper notices
%\IEEEspecialpapernotice{(Invited Paper)}




% make the title area
\maketitle

% As a general rule, do not put math, special symbols or citations
% in the abstract
\begin{abstract}

El NASA Space Apps Challenge 2025 plantea el reto \textit{"Meteor Madness"}, que busca 
desarrollar una herramienta para simular el impacto de asteroides en la Tierra,
evaluando las consecuencias y ayudando a las autoridades pertinentes a tomar 
medidas preventivas. Este proyecto tiene como objetivo crear una página web 
interactiva que permite a los usuarios ingresar parámetros específicos de un 
asteroide, como su tamaño, velocidad y ángulo de entrada, para calcular y 
visualizar las consecuencias medioambientales. La herramienta integra datos 
reales de la NASA Near-Earth Object (NEO) API y de USGS, lo que proporciona 
una base científica sólida para los cálculos. Los resultados incluyen 
estimaciones sobre la extensión del área afectada, la magnitud de la onda 
expansiva, la generación de tsunamis y otros efectos secundarios. Además, se 
incluye una sección educativa que explica los fundamentos científicos detrás de
los impactos de asteroides y las medidas de mitigación. Este proyecto no solo 
tiene un impacto directo en la seguridad planetaria, sino que también sirve 
como un punto de partida para futuras investigaciones en la predicción de 
fenómenos astronómicos peligrosos, apoyándose en datos reales y modelos 
científicos.

\end{abstract}

% no keywords




% For peer review papers, you can put extra information on the cover
% page as needed:
% \ifCLASSOPTIONpeerreview
% \begin{center} \bfseries EDICS Category: 3-BBND \end{center}
% \fi
%
% For peerreview papers, this IEEEtran command inserts a page break and
% creates the second title. It will be ignored for other modes.
\IEEEpeerreviewmaketitle




% REFERENCIAS BIBLIOGRAFICAS ---------------------
% --- Secciones del documento ---
\section{Introducción}

El NASA International Space Apps Challenge 2025, a través del reto \textit{Meteor Madness}, invita a crear soluciones abiertas que ayuden a comprender los efectos de los impactos de meteoroides en la Tierra. En respuesta, este anteproyecto propone un simulador web educativo que permita explorar escenarios de impacto en tierra y en océano, estimando magnitudes clave y visualizando zonas potencialmente afectadas.

El enfoque se apoya en modelos de escalamiento y parametrizaciones ampliamente utilizadas en la literatura para impactos atmosféricos (\textit{airburst}) y formación de cráteres, así como en estimaciones iniciales para la generación de tsunamis por impactos oceánicos \cite{collins2005,holsapple1993,wuennemann2010}. El prototipo prioriza la transparencia (supuestos explícitos, rangos de validez y límites del modelo), la reproducibilidad y la claridad visual, con el objetivo de facilitar tanto la educación como la toma de conciencia sobre el riesgo de impacto.

\section{Objetivos}

Objetivo general: desarrollar un prototipo web de código abierto que estime y visualice los efectos ambientales de impactos de meteoroides, a partir de parámetros físicos básicos y del tipo de blanco (tierra u océano).

Objetivos específicos:
\begin{itemize}
    \item Implementar un núcleo de cálculo con fórmulas paramétricas para energía del impacto, atenuación atmosférica, umbrales de \textit{airburst}/cráter, diámetro de cráter, sobrepresión y carga térmica en función de la distancia, y un esquema inicial para tsunami por impacto oceánico, basados en \cite{collins2005,holsapple1993,wuennemann2010}.
    \item Diseñar una interfaz simple con entradas de diámetro, densidad, velocidad, ángulo de entrada y tipo de blanco; incluir preajustes de ejemplo.
    \item Representar resultados en un mapa o diagrama con anillos concéntricos (isóbaras/energía térmica) y una tabla resumen de magnitudes clave.
    \item Documentar supuestos, rangos de validez e incertidumbres, junto con referencias científicas y fuentes de datos abiertos.
    \item Preparar un conjunto de escenarios de demostración con datos abiertos (por ejemplo, objetos de catálogos de la NASA) y dejar puntos de integración para APIs públicas.
    \item Entregar una demostración funcional, README con instrucciones y un breve video de presentación.
\end{itemize}

\section{Metodología}

\subsection{Entradas y supuestos}
Variables de entrada: diámetro del meteoroide $d$ (m), densidad $\rho_i$ (kg/m$^3$), velocidad a la entrada superior $v$ (m/s), ángulo de entrada $\theta$ (rad), tipo de blanco (tierra u océano), densidad del blanco $\rho_t$, gravedad $g$, resistencia a tracción del meteoroide $\sigma_t$, resistencia del blanco $Y$ y, para océano, profundidad local $h_w$.

Suponemos atmósfera estándar con perfil exponencial: $\rho_a(z) = \rho_0 e^{-z/H}$, con $\rho_0$ densidad al nivel del mar y $H$ la altura de escala. En el MVP, el acoplamiento detallado viento/rotación se omite.

\subsection{Masa y energía}
\begin{align}
    m &= \frac{\pi}{6} \, \rho_i \, d^3, \\
    E_k &= \tfrac{1}{2} m v^2.
\end{align}

\subsection{Entrada atmosférica y ruptura (airburst)}
Modelo simplificado inspirado en \cite{collins2005,chyba1993,popova2013}. El criterio de ruptura por presión dinámica es
\begin{equation}
    q(z) = \tfrac{1}{2} \, \rho_a(z) v^2 \geq \sigma_t.
\end{equation}
Con $\rho_a(z)$ exponencial, la altitud de ruptura aproximada es
\begin{equation}
    z_b \approx H \, \ln\!\left( \frac{\rho_0 v^2}{2\,\sigma_t} \right).
\end{equation}
Si la fragmentación y ablación depositan la mayor parte de la energía por encima del suelo, se considera un \emph{airburst}. Para un tratamiento más completo (arrastre, ablación y cambios de trayectoria) se usan las EDO clásicas de entrada \cite{collins2005,chyba1993}:
\begin{align}
    \frac{dv}{dt} &= -\frac{C_d \, \rho_a A \, v^2}{2 m} - g\,\sin\theta, \\
    \frac{dm}{dt} &= -\frac{C_h \, \rho_a A \, v^3}{2 Q}, \\
    \frac{d\theta}{dt} &= \frac{g\,\cos\theta}{v} - \frac{C_l \, \rho_a A \, v}{2 m},
\end{align}
con $A$ el área frontal, $Q$ el calor de ablación y $C_d, C_h, C_l$ coeficientes aerodinámicos.

\subsection{Craterización: diámetro transitorio y final}
Se emplea el escalamiento $\pi$ de Holsapple y Schmidt \cite{holsapple1993,melosh1989,housen2011}. En régimen gravitacional, una forma práctica (usada en herramientas tipo \cite{collins2005}) es
\begin{equation}
    D_t = K_d \, \left(\frac{\rho_i}{\rho_t}\right)^{1/3} d^{a} v^{b} g^{-c} \, \sin^{1/3}\!\theta,
\end{equation}
con exponentes típicos $a\approx0.78$, $b\approx0.44$, $c\approx0.22$ para blancos rocosos, y $K_d$ dependiente de litología. El diámetro final depende del régimen simple/complejo \cite{melosh1989}:
\begin{align}
    D_f &= K_s \, D_t && (D_t < D_*)\\
    D_f &= K_c \, D_t^{\,\alpha} D_*^{\,1-\alpha} && (D_t \ge D_*)
\end{align}
con $D_*$ el diámetro de transición (\emph{Earth}: $\sim$3--4 km), y $K_s, K_c, \alpha$ constantes empíricas.

El espesor de eyecta decae aproximadamente como ley potencial \cite{mcgetchin1973,housen2011}:
\begin{equation}
    t_{\text{ey}}(r) = K_e \, \rho_t^{-1} \, D_t^{\,\beta} \, r^{-3}, \quad r \ge R_t.
\end{equation}

\subsection{Sobrepresión de onda de choque}
Para \emph{airbursts} o explosiones superficiales se usa escalamiento por energía equivalente TNT y la distancia escalada $Z = R/W^{1/3}$, con $W = \eta_p E_k$ y eficiencia $\eta_p$ \cite{kingery1984,ufc334002,collins2005}. La sobrepresión $\Delta P$ se obtiene de curvas de Kingery--Bulmash o de ajustes polinomiales estándar:
\begin{equation}
    \Delta P = f(Z), \quad Z = \frac{R}{W^{1/3}}.
\end{equation}
Se reportan radios para umbrales típicos (p.ej., 1, 3, 5, 10 psi) para visualización.

\subsection{Radiación térmica}
Se adopta un modelo de \emph{fireball} con eficiencia radiante $\eta_r$ \cite{glasstone1977,collins2005}. La fluencia térmica a distancia $R$ es
\begin{equation}
    F(R) = \eta_r \, \frac{E_k}{4\pi R^2} \, T_{\text{atm}}(R,\,z_b) \, \cos\iota,
\end{equation}
con $T_{\text{atm}}$ transmisión atmosférica efectiva y $\iota$ el ángulo de incidencia. Se comparan umbrales de efectos térmicos (p.ej., ignición de vegetación, quemaduras de 1./2./3. grado) \cite{glasstone1977}.

\subsection{Tsunami por impacto oceánico}
Para impactos en océano, la amplitud inicial $H_0$ se relaciona con la cavidad transitoria o con la energía acoplada en el agua \cite{ward2000,wuennemann2010}:
\begin{equation}
    H_0 \propto \left( \frac{E_w}{\rho_w g} \right)^{1/4}, \quad E_w = \eta_w E_k.
\end{equation}
La propagación se aproxima por atenuación geométrica y disipación:
\begin{equation}
    H(r) = H_0 \left( \frac{r_0}{r} \right)^p e^{-\kappa r}, \quad p\in[\tfrac{1}{2},1],
\end{equation}
ajustando $p$ al régimen próximo/lejos de la fuente y $\kappa$ a pérdidas dispersivas y de fondo.

\subsection{Salidas del prototipo}
Se calculan y visualizan: energía depositada y altitud de ruptura; diámetros $D_t$ y $D_f$ (si aplica); radios para umbrales de sobrepresión y de fluencia térmica; espesor de eyecta estimado; y, en océano, amplitudes $H(r)$ en anillos concéntricos. Cada resultado se acompaña de supuestos y rangos de validez.

\section{Datos abiertos NASA}

\section{Arquitectura y alcance}

\section{Plan de trabajo (48 h) y entregables}

\section{Riesgos y mitigaciones}

% --- Fin de secciones ---
% (Contenido por completar en cada sección.)

% REFERENCIAS BIBLIOGRAFICAS ---------------------
\begin{thebibliography}{00}

    \bibitem{collins2005}
    G. S. Collins, H. J. Melosh, and R. A. Marcus, ``Earth impact effects program: A web-based computer program for calculating the regional environmental consequences of a meteoroid impact on earth,'' \textit{Meteoritics \& Planetary Science}, vol. 40, no. 6, pp. 817--840, 2005. doi:10.1111/j.1945-5100.2005.tb00157.x

    \bibitem{herrick2006}
    R. R. Herrick, ``Updates regarding the resurfacing of venusian impact craters,'' in \textit{Lunar and Planet. Sci. Conf. XXXVII}, p. Abs. 1588, Lunar and Planetary Institute, Houston, Texas, 2006.

    \bibitem{herrick1997}
    R. R. Herrick, V. L. Sharpton, M. C. Malin, S. N. Lyons, and K. Feely, ``Morphology and Morphometry of Impact Craters,'' in \textit{Venus II: Geology, Geophysics, Atmosphere, and Solar Wind Environment}, S. W. Bougher, D. M. Hunten, and R. J. Phillips, Eds., p. 1015, 1997.

    \bibitem{holsapple1993}
    K. A. Holsapple, ``The scaling of impact processes in planetary sciences,'' \textit{Ann. Rev. Earth Planet. Sci.}, vol. 21, pp. 333--373, 1993.

    \bibitem{mckinnon1985}
    W. B. McKinnon and P. M. Schenk, ``Ejecta blanket scaling on the Moon and Mercury - inferences for projectile populations,'' in \textit{Lunar and Planet. Sci. Conf. Proceedings XVI}, pp. 544--545, Lunar and Planetary Institute, Houston, Texas, 1985.

    \bibitem{wuennemann2010}
    K. Wünnemann, G. S. Collins, and R. Weiss, ``Impact of a cosmic body into earth’s ocean and the generation of large tsunami waves: Insight from numerical modeling,'' \textit{Reviews of Geophysics}, vol. 48, no. 4, 2010. doi:10.1029/2009RG000308
    
    \bibitem{melosh1989}
    H. J. Melosh, \textit{Impact Cratering: A Geologic Process}. Oxford University Press, 1989.

    \bibitem{housen2011}
    K. R. Housen and K. A. Holsapple, ``Ejecta from impact craters,'' \textit{Icarus}, vol. 211, pp. 856--875, 2011. doi:10.1016/j.icarus.2010.09.017

    \bibitem{mcgetchin1973}
    T. R. McGetchin, M. Settle, and J. W. Head, ``Lunar impact ejection and crater growth,'' \textit{Journal of Geophysical Research}, vol. 78, no. 11, pp. 10847--10863, 1973. doi:10.1029/JB078i023p10847

    \bibitem{chyba1993}
    C. F. Chyba, P. J. Thomas, and K. J. Zahnle, ``The 1908 Tunguska explosion: Atmospheric disruption of a stony asteroid,'' \textit{Nature}, vol. 361, pp. 40--44, 1993. doi:10.1038/361040a0

    \bibitem{popova2013}
    O. P. Popova et al., ``Chelyabinsk Airburst, Damage Assessment, Meteorite Recovery, and Characterization,'' \textit{Science}, vol. 342, no. 6162, pp. 1069--1073, 2013. doi:10.1126/science.1242642

    \bibitem{kingery1984}
    C. N. Kingery and G. Bulmash, ``Airblast Parameters from TNT Spherical Air Burst and Hemispherical Surface Burst,'' Technical Report ARBRL-TR-02555, U.S. Army Ballistic Research Laboratory, Aberdeen Proving Ground, 1984.

    \bibitem{ufc334002}
    UFC 3-340-02, ``Structures to Resist the Effects of Accidental Explosions,'' U.S. Department of Defense, 2008 (Change 2, 2014).

    \bibitem{glasstone1977}
    S. Glasstone and P. J. Dolan, \textit{The Effects of Nuclear Weapons}. U.S. Department of Defense and U.S. Department of Energy, 1977.

    \bibitem{ward2000}
    S. N. Ward and E. Asphaug, ``Asteroid impact tsunami: A probabilistic hazard assessment,'' \textit{Icarus}, vol. 145, pp. 64--78, 2000. doi:10.1006/icar.1999.6336

\end{thebibliography}



\end{document}
