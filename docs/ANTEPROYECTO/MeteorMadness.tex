%% bare_conf.tex
%% V1.4b
%% 2015/08/26
%% by Michael Shell
%% See:
%% http://www.michaelshell.org/
%% for current contact information.
%%
%% This is a skeleton file demonstrating the use of IEEEtran.cls
%% (requires IEEEtran.cls version 1.8b or later) with an IEEE
%% conference paper.
%%
%% Support sites:
%% http://www.michaelshell.org/tex/ieeetran/
%% http://www.ctan.org/pkg/ieeetran
%% and
%% http://www.ieee.org/

%%*************************************************************************
%% Legal Notice:
%% This code is offered as-is without any warranty either expressed or
%% implied; without even the implied warranty of MERCHANTABILITY or
%% FITNESS FOR A PARTICULAR PURPOSE! 
%% User assumes all risk.
%% In no event shall the IEEE or any contributor to this code be liable for
%% any damages or losses, including, but not limited to, incidental,
%% consequential, or any other damages, resulting from the use or misuse
%% of any information contained here.
%%
%% All comments are the opinions of their respective authors and are not
%% necessarily endorsed by the IEEE.
%%
%% This work is distributed under the LaTeX Project Public License (LPPL)
%% ( http://www.latex-project.org/ ) version 1.3, and may be freely used,
%% distributed and modified. A copy of the LPPL, version 1.3, is included
%% in the base LaTeX documentation of all distributions of LaTeX released
%% 2003/12/01 or later.
%% Retain all contribution notices and credits.
%% ** Modified files should be clearly indicated as such, including  **
%% ** renaming them and changing author support contact information. **
%%*************************************************************************


% *** Authors should verify (and, if needed, correct) their LaTeX system  ***
% *** with the testflow diagnostic prior to trusting their LaTeX platform ***
% *** with production work. The IEEE's font choices and paper sizes can   ***
% *** trigger bugs that do not appear when using other class files.       ***                          ***
% The testflow support page is at:
% http://www.michaelshell.org/tex/testflow/



\documentclass[conference]{IEEEtran}
\usepackage{cite}
\usepackage{graphicx}
\usepackage{hyperref}
\usepackage{float}

% Some Computer Society conferences also require the compsoc mode option,
% but others use the standard conference format.
%
% If IEEEtran.cls has not been installed into the LaTeX system files,
% manually specify the path to it like:
% \documentclass[conference]{../sty/IEEEtran}





% Some very useful LaTeX packages include:
% (uncomment the ones you want to load)


% *** MISC UTILITY PACKAGES ***
%
%\usepackage{ifpdf}
% Heiko Oberdiek's ifpdf.sty is very useful if you need conditional
% compilation based on whether the output is pdf or dvi.
% usage:
% \ifpdf
%   % pdf code
% \else
%   % dvi code
% \fi
% The latest version of ifpdf.sty can be obtained from:
% http://www.ctan.org/pkg/ifpdf
% Also, note that IEEEtran.cls V1.7 and later provides a builtin
% \ifCLASSINFOpdf conditional that works the same way.
% When switching from latex to pdflatex and vice-versa, the compiler may
% have to be run twice to clear warning/error messages.






% *** CITATION PACKAGES ***
%
%\usepackage{cite}
% cite.sty was written by Donald Arseneau
% V1.6 and later of IEEEtran pre-defines the format of the cite.sty package
% \cite{} output to follow that of the IEEE. Loading the cite package will
% result in citation numbers being automatically sorted and properly
% "compressed/ranged". e.g., [1], [9], [2], [7], [5], [6] without using
% cite.sty will become [1], [2], [5]--[7], [9] using cite.sty. cite.sty's
% \cite will automatically add leading space, if needed. Use cite.sty's
% noadjust option (cite.sty V3.8 and later) if you want to turn this off
% such as if a citation ever needs to be enclosed in parenthesis.
% cite.sty is already installed on most LaTeX systems. Be sure and use
% version 5.0 (2009-03-20) and later if using hyperref.sty.
% The latest version can be obtained at:
% http://www.ctan.org/pkg/cite
% The documentation is contained in the cite.sty file itself.






% *** GRAPHICS RELATED PACKAGES ***
%
\ifCLASSINFOpdf
  % \usepackage[pdftex]{graphicx}
  % declare the path(s) where your graphic files are
  % \graphicspath{{../pdf/}{../jpeg/}}
  % and their extensions so you won't have to specify these with
  % every instance of \includegraphics
  % \DeclareGraphicsExtensions{.pdf,.jpeg,.png}
\else
  % or other class option (dvipsone, dvipdf, if not using dvips). graphicx
  % will default to the driver specified in the system graphics.cfg if no
  % driver is specified.
  % \usepackage[dvips]{graphicx}
  % declare the path(s) where your graphic files are
  % \graphicspath{{../eps/}}
  % and their extensions so you won't have to specify these with
  % every instance of \includegraphics
  % \DeclareGraphicsExtensions{.eps}
\fi
% graphicx was written by David Carlisle and Sebastian Rahtz. It is
% required if you want graphics, photos, etc. graphicx.sty is already
% installed on most LaTeX systems. The latest version and documentation
% can be obtained at: 
% http://www.ctan.org/pkg/graphicx
% Another good source of documentation is "Using Imported Graphics in
% LaTeX2e" by Keith Reckdahl which can be found at:
% http://www.ctan.org/pkg/epslatex
%
% latex, and pdflatex in dvi mode, support graphics in encapsulated
% postscript (.eps) format. pdflatex in pdf mode supports graphics
% in .pdf, .jpeg, .png and .mps (metapost) formats. Users should ensure
% that all non-photo figures use a vector format (.eps, .pdf, .mps) and
% not a bitmapped formats (.jpeg, .png). The IEEE frowns on bitmapped formats
% which can result in "jaggedy"/blurry rendering of lines and letters as
% well as large increases in file sizes.
%
% You can find documentation about the pdfTeX application at:
% http://www.tug.org/applications/pdftex





% *** MATH PACKAGES ***
%
%\usepackage{amsmath}
% A popular package from the American Mathematical Society that provides
% many useful and powerful commands for dealing with mathematics.
%
% Note that the amsmath package sets \interdisplaylinepenalty to 10000
% thus preventing page breaks from occurring within multiline equations. Use:
%\interdisplaylinepenalty=2500
% after loading amsmath to restore such page breaks as IEEEtran.cls normally
% does. amsmath.sty is already installed on most LaTeX systems. The latest
% version and documentation can be obtained at:
% http://www.ctan.org/pkg/amsmath





% *** SPECIALIZED LIST PACKAGES ***
%
%\usepackage{algorithmic}
% algorithmic.sty was written by Peter Williams and Rogerio Brito.
% This package provides an algorithmic environment fo describing algorithms.
% You can use the algorithmic environment in-text or within a figure
% environment to provide for a floating algorithm. Do NOT use the algorithm
% floating environment provided by algorithm.sty (by the same authors) or
% algorithm2e.sty (by Christophe Fiorio) as the IEEE does not use dedicated
% algorithm float types and packages that provide these will not provide
% correct IEEE style captions. The latest version and documentation of
% algorithmic.sty can be obtained at:
% http://www.ctan.org/pkg/algorithms
% Also of interest may be the (relatively newer and more customizable)
% algorithmicx.sty package by Szasz Janos:
% http://www.ctan.org/pkg/algorithmicx




% *** ALIGNMENT PACKAGES ***
%
%\usepackage{array}
% Frank Mittelbach's and David Carlisle's array.sty patches and improves
% the standard LaTeX2e array and tabular environments to provide better
% appearance and additional user controls. As the default LaTeX2e table
% generation code is lacking to the point of almost being broken with
% respect to the quality of the end results, all users are strongly
% advised to use an enhanced (at the very least that provided by array.sty)
% set of table tools. array.sty is already installed on most systems. The
% latest version and documentation can be obtained at:
% http://www.ctan.org/pkg/array


% IEEEtran contains the IEEEeqnarray family of commands that can be used to
% generate multiline equations as well as matrices, tables, etc., of high
% quality.




% *** SUBFIGURE PACKAGES ***
%\ifCLASSOPTIONcompsoc
%  \usepackage[caption=false,font=normalsize,labelfont=sf,textfont=sf]{subfig}
%\else
%  \usepackage[caption=false,font=footnotesize]{subfig}
%\fi
% subfig.sty, written by Steven Douglas Cochran, is the modern replacement
% for subfigure.sty, the latter of which is no longer maintained and is
% incompatible with some LaTeX packages including fixltx2e. However,
% subfig.sty requires and automatically loads Axel Sommerfeldt's caption.sty
% which will override IEEEtran.cls' handling of captions and this will result
% in non-IEEE style figure/table captions. To prevent this problem, be sure
% and invoke subfig.sty's "caption=false" package option (available since
% subfig.sty version 1.3, 2005/06/28) as this is will preserve IEEEtran.cls
% handling of captions.
% Note that the Computer Society format requires a larger sans serif font
% than the serif footnote size font used in traditional IEEE formatting
% and thus the need to invoke different subfig.sty package options depending
% on whether compsoc mode has been enabled.
%
% The latest version and documentation of subfig.sty can be obtained at:
% http://www.ctan.org/pkg/subfig




% *** FLOAT PACKAGES ***
%
%\usepackage{fixltx2e}
% fixltx2e, the successor to the earlier fix2col.sty, was written by
% Frank Mittelbach and David Carlisle. This package corrects a few problems
% in the LaTeX2e kernel, the most notable of which is that in current
% LaTeX2e releases, the ordering of single and double column floats is not
% guaranteed to be preserved. Thus, an unpatched LaTeX2e can allow a
% single column figure to be placed prior to an earlier double column
% figure.
% Be aware that LaTeX2e kernels dated 2015 and later have fixltx2e.sty's
% corrections already built into the system in which case a warning will
% be issued if an attempt is made to load fixltx2e.sty as it is no longer
% needed.
% The latest version and documentation can be found at:
% http://www.ctan.org/pkg/fixltx2e


%\usepackage{stfloats}
% stfloats.sty was written by Sigitas Tolusis. This package gives LaTeX2e
% the ability to do double column floats at the bottom of the page as well
% as the top. (e.g., "\begin{figure*}[!b]" is not normally possible in
% LaTeX2e). It also provides a command:
%\fnbelowfloat
% to enable the placement of footnotes below bottom floats (the standard
% LaTeX2e kernel puts them above bottom floats). This is an invasive package
% which rewrites many portions of the LaTeX2e float routines. It may not work
% with other packages that modify the LaTeX2e float routines. The latest
% version and documentation can be obtained at:
% http://www.ctan.org/pkg/stfloats
% Do not use the stfloats baselinefloat ability as the IEEE does not allow
% \baselineskip to stretch. Authors submitting work to the IEEE should note
% that the IEEE rarely uses double column equations and that authors should try
% to avoid such use. Do not be tempted to use the cuted.sty or midfloat.sty
% packages (also by Sigitas Tolusis) as the IEEE does not format its papers in
% such ways.
% Do not attempt to use stfloats with fixltx2e as they are incompatible.
% Instead, use Morten Hogholm'a dblfloatfix which combines the features
% of both fixltx2e and stfloats:
%
% \usepackage{dblfloatfix}
% The latest version can be found at:
% http://www.ctan.org/pkg/dblfloatfix




% *** PDF, URL AND HYPERLINK PACKAGES ***
%
%\usepackage{url}
% url.sty was written by Donald Arseneau. It provides better support for
% handling and breaking URLs. url.sty is already installed on most LaTeX
% systems. The latest version and documentation can be obtained at:
% http://www.ctan.org/pkg/url
% Basically, \url{my_url_here}.




% *** Do not adjust lengths that control margins, column widths, etc. ***
% *** Do not use packages that alter fonts (such as pslatex).         ***
% There should be no need to do such things with IEEEtran.cls V1.6 and later.
% (Unless specifically asked to do so by the journal or conference you plan
% to submit to, of course. )


% correct bad hyphenation here
\hyphenation{op-tical net-works semi-conduc-tor}
\usepackage{listings}
\usepackage{xcolor}
\usepackage{algorithm}
\usepackage{algpseudocode}
\usepackage{amsmath}

\begin{document}
%
% paper title
% Titles are generally capitalized except for words such as a, an, and, as,
% at, but, by, for, in, nor, of, on, or, the, to and up, which are usually
% not capitalized unless they are the first or last word of the title.
% Linebreaks \\ can be used within to get better formatting as desired.
% Do not put math or special symbols in the title.
\title{SIAER(Simulador de Impactos de Asteroides y Evaluación de Riesgos): Una
	herramienta web para calcular las consecuencias medioambientales de impactos de
	asteroides en la Tierra}


% author names and affiliations
% use a multiple column layout for up to three different
% affiliations
\author{
	\IEEEauthorblockN{Bendezu Pastrana, Tommy}
	\IEEEauthorblockA{Dept. de Ing. Mecatrónica\\
		Universidad Continental\\
		Huancayo, Perú\\
		\href{mailto:tommyhanss007@gmail.com}{tommyhanss007@gmail.com}}

	\and

	\IEEEauthorblockN{Meza Millan, Martin}
	\IEEEauthorblockA{Dept. de Ing. Civil\\
		Universidad Continental\\
		Huancayo, Perú\\
		\href{mailto:mezamilland@gmail.com}{mezamilland@gmail.com}}

	\and

	\IEEEauthorblockN{Ildefonso Santos, Steve}
	\IEEEauthorblockA{Dept. de Computer Science\\
		UTEC\\
		Huancayo, Perú\\
		\href{mailto:steve.ildefonso@utec.edu.pe}{steve.ildefonso@utec.edu.pe}}

	\and

	\IEEEauthorblockN{Carhuanta Piélago, Mijael}
	\IEEEauthorblockA{Huancayo, Perú\\
		\href{mailto:carhuamantaanyelo159753@gmail.com}{carhuamantaanyelo159753@gmail.com}}
}

% conference papers do not typically use \thanks and this command
% is locked out in conference mode. If really needed, such as for
% the acknowledgment of grants, issue a \IEEEoverridecommandlockouts
% after \documentclass

% for over three affiliations, or if they all won't fit within the width
% of the page, use this alternative format:
% 
%\author{\IEEEauthorblockN{Michael Shell\IEEEauthorrefmark{1},
%Homer Simpson\IEEEauthorrefmark{2},
%James Kirk\IEEEauthorrefmark{3}, 
%Montgomery Scott\IEEEauthorrefmark{3} and
%Eldon Tyrell\IEEEauthorrefmark{4}}
%\IEEEauthorblockA{\IEEEauthorrefmark{1}School of Electrical and Computer Engineering\\
%Georgia Institute of Technology,
%Atlanta, Georgia 30332--0250\\ Email: see http://www.michaelshell.org/contact.html}
%\IEEEauthorblockA{\IEEEauthorrefmark{2}Twentieth Century Fox, Springfield, USA\\
%Email: homer@thesimpsons.com}
%\IEEEauthorblockA{\IEEEauthorrefmark{3}Starfleet Academy, San Francisco, California 96678-2391\\
%Telephone: (800) 555--1212, Fax: (888) 555--1212}
%\IEEEauthorblockA{\IEEEauthorrefmark{4}Tyrell Inc., 123 Replicant Street, Los Angeles, California 90210--4321}}




% use for special paper notices
%\IEEEspecialpapernotice{(Invited Paper)}




% make the title area
\maketitle

% As a general rule, do not put math, special symbols or citations
% in the abstract
\begin{abstract}

	El NASA Space Apps Challenge 2025 plantea el reto \textit{"Meteor Madness"}, que busca
	desarrollar una herramienta para simular el impacto de asteroides en la Tierra,
	evaluando las consecuencias y ayudando a las autoridades pertinentes a tomar
	medidas preventivas. Este proyecto tiene como objetivo crear una página web
	interactiva que permite a los usuarios ingresar parámetros específicos de un
	asteroide, como su tamaño, velocidad y ángulo de entrada, para calcular y
	visualizar las consecuencias medioambientales. La herramienta integra datos
	reales de la NASA Near-Earth Object (NEO) API y de USGS, lo que proporciona
	una base científica sólida para los cálculos. Los resultados incluyen
	estimaciones sobre la extensión del área afectada, la magnitud de la onda
	expansiva, la generación de tsunamis y otros efectos secundarios. Además, se
	incluye una sección educativa que explica los fundamentos científicos detrás de
	los impactos de asteroides y las medidas de mitigación. Este proyecto no solo
	tiene un impacto directo en la seguridad planetaria, sino que también sirve
	como un punto de partida para futuras investigaciones en la predicción de
	fenómenos astronómicos peligrosos, apoyándose en datos reales y modelos
	científicos.

\end{abstract}

% no keywords




% For peer review papers, you can put extra information on the cover
% page as needed:
% \ifCLASSOPTIONpeerreview
% \begin{center} \bfseries EDICS Category: 3-BBND \end{center}
% \fi
%
% For peerreview papers, this IEEEtran command inserts a page break and
% creates the second title. It will be ignored for other modes.
\IEEEpeerreviewmaketitle

% --- INTRODUCCION ---
\section{Introducción}

El NASA Space Apps Challenge 2025, a través del reto \textit{Meteor Madness},
invita a la comunidad global a desarrollar soluciones innovadoras para entender
y mitigar los riesgos que representan los asteroides cercanos a la Tierra.
Aunque los impactos de asteroides son eventos poco frecuentes, su potencial
destructivo es significativo, con efectos que incluyen tsunamis, terremotos y
alteraciones atmosféricas. A lo largo de la historia, impactos de este tipo han

tenido consecuencias devastadoras, como el evento de Chicxulub, que se cree que
contribuyó a la extinción de los dinosaurios. Esta amenaza subraya la
importancia de contar con herramientas que nos permitan comprender y evaluar
estos riesgos \cite{ward2000,melosh1989}.

Las herramientas interactivas de simulación juegan un papel crucial en la visualización de estos riesgos, ya que permiten explorar diferentes escenarios de impacto y sus consecuencias medioambientales. Estas herramientas no solo ayudan a concienciar al público y a los responsables de la toma de decisiones sobre la magnitud de los impactos, sino que también permiten probar estrategias de mitigación, como la desviación de asteroides. La capacidad de simular cómo los cambios en los parámetros de un asteroide (tamaño, velocidad, trayectoria) afectan el impacto puede proporcionar información valiosa para la planificación de la protección planetaria \cite{collins2005}.

En respuesta a este reto, este anteproyecto propone el desarrollo de un simulador web educativo que permite calcular y visualizar las consecuencias medioambientales de su impacto. Integrando datos reales de la NASA Near-Earth Object (NEO) API y del USGS, el simulador proporcionará estimaciones sobre la extensión del área afectada, la magnitud de la onda expansiva, la formación de cráteres y la generación de tsunamis, todo basado en modelos científicos establecidos. Además, el simulador incluirá una sección educativa que explicará los fundamentos científicos detrás de los impactos de asteroides y las medidas de mitigación \cite{collins2005,wuennemann2010}.


% --- OBJETIVOS ---
\section{Objetivos}

\textbf{Objetivo general}: Desarrollar una herramienta web interactiva que
simule el impacto de asteroides cercanos a la Tierra, permitiendo a los usuarios
visualizar las consecuencias medioambientales y explorar estrategias de
mitigación, utilizando datos reales de la \textit{NASA} y el \textit{USGS}.

\textbf{Objetivos específicos}:
\begin{enumerate}
	\item \textbf{Simular la trayectoria de asteroides cercanos a la Tierra}:
	      \begin{itemize}
		      \item{ Implementar un modelo orbital que permita calcular y
		            visualizar la trayectoria de un asteroide basándose en
		            parámetros Keplerianos.}
	      \end{itemize}

	\item \textbf{Visualizar los efectos medioambientales de un impacto}:
	      \begin{itemize}
		      \item{Calcular, mediante modelos científicos, y mostrar las
		            consecuencias de un impacto de asteroide, tales como la
		            formación de cráteres, la generación de tsunamis, la actividad
		            sísmica asociada, utilizando datos y modelos científicos existentes.}
	      \end{itemize}

	\item \textbf{Integrar datos reales de la \textit{NASA} y el \textit{USGS}}:
	      \begin{itemize}
		      \item{Utilizar la NASA Near-Earth Object (NEO) API para obtener
		            información sobre asteroides cercanos y la USGS Earthquake
		            Catalog para modelar los efectos sísmicos, generando así una
		            simulación realista y precisa.}
	      \end{itemize}

	\item \textbf{Incluir una sección educativa sobre impactos de asteroides}:
	      \begin{itemize}

		      \item{Desarrollar una sección educativa dentro de la herramienta
		            que explique los fundamentos científicos detrás de los
		            impactos de asteroides, las posibles consecuencias
		            medioambientales y las estrategias de mitigación.}
	      \end{itemize}
	\item\textbf{Proveer una interfaz interactiva y accesible para usuarios no expertos}:
	      \begin{itemize}
		      \item{Crear una interfaz amigable que permita a los usuarios
		            modificar parámetros (como el tamaño y la velocidad del
		            asteroide) y ver los resultados en tiempo real, asegurando que la
		            herramienta sea accesible tanto para científicos como para el
		            público general.}


	      \end{itemize}

	\item\textbf{Evaluar y visualizar estrategias de mitigación de impactos}:
	      \begin{itemize}
		      \item{Implementar la simulación de estrategias de mitigación,
		            como la desviación de asteroides, y permitir a los usuarios
		            visualizar cómo estas estrategias afectan la trayectoria del
		            asteroide y los efectos del impacto.}

	      \end{itemize}
\end{enumerate}

% --- METODOLOGIA ---
\section{Metodología}
\subsection{Enfoque del Proyecto}
El proyecto seguirá un enfoque \textbf{iterativo} dividido en varias fases
clave, que se llevarán a cabo de forma paralela cuando sea posible, para
maximizar la eficiencia. Usando la herramienta \textbf{GitHub} tendremos una
evolución paralela en cada aspecto de la app(Frontend, Backend, etc) y
luego se integrarán en un solo producto.

\subsection{Desarrollo de la Simulación}
La simulación se desarrollará en dos partes principales: la \textbf{trayectoria
	de asteroide} y los \textbf{efectos del impacto}.
\begin{itemize}
	\item \textbf{Simulación de la Trayectoria}: Basándonos en la metodología de
	      diseño de órbitas elípticas de NASA Mission Visualization,
	      se definirán los elementos orbitales del
	      asteroide (semi-eje mayor, excentricidad, inclinación, longitud del nodo
	      ascendente y argumento del periastro) para inicializar la simulación. La
	      propagación temporal se realizará resolviendo la ecuación de Kepler,
	      \begin{align}
		      M(t) & = n\,(t - t_0) = E - e \sin E, \label{eq:kepler}
	      \end{align}
	      para obtener la anomalía excéntrica $E$. La anomalía verdadera $\nu$ se
	      calculará mediante
	      \begin{align}
		      \tan\frac{\nu}{2} & = \sqrt{\frac{1 + e}{1 - e}}\,\tan\frac{E}{2},
	      \end{align}
	      mientras que la distancia al foco se deriva de
	      \begin{align}
		      r & = a\,(1 - e \cos E).
	      \end{align}
	      Con estas magnitudes se evaluará el vector de posición en un marco
	      centrado en la Tierra aplicando las rotaciones de Euler sobre el plano de
	      la órbita,
	      \begin{align}
		      \mathbf{r}_{\text{ECI}} & =
		      R_3(-\Omega)\,R_1(-i)\,R_3(-\omega)
		      \begin{bmatrix}
			      r \cos \nu \\
			      r \sin \nu \\
			      0
		      \end{bmatrix},
	      \end{align}
	      lo que permitirá calcular la posición y la velocidad en un marco centrado
	      en la Tierra. Con estas magnitudes se generará la trayectoria en tres
	      dimensiones y se estimará el punto de impacto considerando la rotación
	      terrestre, la velocidad de llegada y diferentes ángulos de entrada.
	      \cite{nasaEllipticalOrbit}
	\item \textbf{Simulación de los Efectos del Impacto}: Se seguirá el flujo
	      establecido en el Earth Impact Effects Program \cite{collins2005},
	      complementado con las referencias de ingeniería de explosiones. Las
	      propiedades iniciales del proyectil se definen como
	      \begin{align}
		      m   & = \tfrac{\pi}{6}\,\rho_i D_i^3,          &
		      R_i & = \tfrac{D_i}{2}, \label{eq:impact-mass}
	      \end{align}
	      y la evolución atmosférica se integra resolviendo las ecuaciones de
	      deceleración y ablación \cite{collins2005},
	      \begin{align}
		      \frac{dv}{dt} & = -\frac{C_D A \rho_{\text{atm}}(h)}{2 m}\,v^2 - g\sin\gamma, \label{eq:drag} \\
		      \frac{dm}{dt} & = -\frac{C_h A \rho_{\text{atm}}(h)}{2 Q^*}\,v^3, \label{eq:ablation}
	      \end{align}
	      evaluando la ruptura cuando la presión dinámica supera la resistencia del
	      material siguiendo
	      \begin{align}
		      q(h) & = \tfrac{1}{2}\,\rho_{\text{atm}}(h)\,v^2 \ge \sigma_c. \label{eq:breakup}
	      \end{align}
	      Si el cuerpo alcanza el suelo, se calcula la energía cinética y su
	      equivalente en kilotones de TNT
	      \begin{align}
		      E_k           & = \tfrac{1}{2} m v_{\text{impacto}}^2,                          &
		      W_{\text{kt}} & = \frac{E_k}{4.184 \times 10^{12}\ \text{J}}, \label{eq:energy}
	      \end{align}
	      distribuyendo la energía entre fracciones sísmica, térmica y de onda de
	      choque según las eficiencias \cite{collins2005,glasstone1977}. Para el
	      dimensionamiento del cráter se aplican las leyes de escala
	      pi-adimensionales de Holsapple adoptadas por \cite{collins2005}:
	      \begin{align}
		      \pi_2 & = \frac{g R_i}{v_{\text{impacto}}^2},                           &
		      \pi_3 & = \frac{Y_t}{\rho_t v_{\text{impacto}}^2}, \label{eq:pi-groups}
	      \end{align}
	      con $Y_t$ la resistencia del objetivo. El radio del cráter transitorio se
	      obtiene tomando el máximo entre los regímenes de gravedad y de resistencia
	      \cite{collins2005,holsapple1993},
	      \begin{align}
		      R_t^{(g)} & = K_g R_i \Bigl(\frac{\rho_i}{\rho_t}\Bigr)^{1/3}\pi_2^{-\mu_g}, \label{eq:rt-gravity}  \\
		      R_t^{(s)} & = K_s R_i \Bigl(\frac{\rho_i}{\rho_t}\Bigr)^{1/3}\pi_3^{-\mu_s}, \label{eq:rt-strength} \\
		      R_t       & = \max\bigl(R_t^{(g)}, R_t^{(s)}\bigr), \label{eq:rt-final}
	      \end{align}
	      con coeficientes $K_g = 1.161$, $\mu_g = 0.22$, $K_s = 1.03$ y
	      $\mu_s = 0.275$ para objetivos rocosos. El diámetro final distingue entre
	      cráter simple y complejo \cite{collins2005,herrick1997},
	      \begin{align}
		      D_f & =
		      \begin{cases}
			      1.25\,D_t,                              & D_t \le D_{\text{sc}}, \\
			      1.17\,D_t^{1.13} D_{\text{sc}}^{-0.13}, & D_t > D_{\text{sc}},
		      \end{cases}\label{eq:final-crater}
	      \end{align}
	      donde $D_t = 2 R_t$ y $D_{\text{sc}}$ es el umbral local de transición
	      simple-complejo. La eyección balística se modelará con las leyes de
	      decaimiento radial del programa \cite{collins2005,herrick2006},
	      \begin{align}
		      h_e(r) & = 0.14 R_t \Bigl(\frac{R_t}{r}\Bigr)^3, \label{eq:ejecta-thickness} \\
		      v_e(r) & = \sqrt{2 g R_t}\Bigl(\frac{R_t}{r}\Bigr)^{3/2},
	      \end{align}
	      que permiten estimar espesores de manta y velocidades de deposición. Para
	      los efectos atmosféricos se adopta la eficiencia de acoplamiento explosivo
	      $\eta_b$ propuesta en \cite{collins2005} y se emplea el escalado de
	      Kingery--Bulmash \cite{ufc334002} sobre la carga equivalente
	      $W_{\text{eq}} = \eta_b W_{\text{kt}}$,
	      \begin{align}
		      Z           & = \frac{R}{W_{\text{eq}}^{1/3}},                  \\
		      \Delta P(Z) & = \exp\Biggl(\sum_{i=0}^{5} a_i (\ln Z)^i\Biggr), \\
		      I(Z)        & = \exp\Biggl(\sum_{i=0}^{5} b_i (\ln Z)^i\Biggr)
	      \end{align}
	      evaluando la carga dinámica estructural mediante
	      \begin{align}
		      q_{\text{din}} & = \Delta P(Z) + \tfrac{1}{2} \rho_{\text{aire}} U_s^2(Z),
	      \end{align}
	      donde $U_s(Z)$ proviene de la relación de Rankine--Hugoniot
	      \cite{ufc334002}. La componente térmica utiliza la eficiencia luminosa
	      $\eta_L$ tabulada en \cite{collins2005},
	      \begin{align}
		      H(R) & = \eta_L\,\frac{E_k}{4 \pi R^2 \tau},
	      \end{align}
	      integrando la exposición sobre la duración $\tau$ para derivar dosis
	      por estadio. En impactos oceánicos se conserva la modelación de cavidad y
	      tsunami de Ward y Asphaug \cite{ward2000},
	      \begin{align}
		      r_c     & = \Biggl(\frac{3 E_k}{2 \pi \rho_w g}\Biggr)^{1/3}, &
		      \eta(r) & = \eta_0 \frac{r_c}{r}\,
		      \exp\Bigl(-\frac{r - r_c}{L_d}\Bigr),
	      \end{align}
	      propagando el \textit{run-up} costero mediante la ley de Green,
	      \begin{align}
		      R_{\text{run-up}} & = \eta(r_s)
		      \Biggl(\frac{h_s}{h(r_s)}\Biggr)^{1/4}.
	      \end{align}
	      Esta cadena de modelos cubre los escenarios de estallido atmosférico,
	      impacto terrestre u oceánico y entrega mapas de daño térmico, estructural
	      e inundación consistentes con las referencias empleadas.
\end{itemize}

\subsection{Interactividad y Visualización}
La interfaz se diseñará para permitir que el usuario ingrese, ajuste y
visualice los parámetros del asteroide y los resultados de la simulación en
tiempo real.
\begin{itemize}
	\item \textbf{Entrada de Parámetros}: Los usuarios podrán modificar parámetros
	      como \textbf{tamaño}, \textbf{velocidad}, \textbf{ángulo de entrada}, mediante
	      \textbf{sliders} y \textbf{campos de texto}.
	\item \textbf{Visualización en 3D}: Se utilizará \textbf{Three.js} para
	      renderizar la \textbf{trayectoria del asteroide} en un entorno 3D
	      interactivo, permitiendo rotar, acercar y alejar la vista.
	\item \textbf{Visualización en 2D}: Se empleará \textbf{D3.js} para crear gráficos y mapas que
	      representen los \textbf{efectos del impacto}, como la extensión del área afectada,
	      la magnitud de la onda expansiva y la generación de tsunamis.
	\item \textbf{Actualización en tiempo real}: Los resultados de la simulación se
	      actualizarán dinámicamente a medida que el usuario modifique los parámetros,
	      proporcionando retroalimentación inmediata.
	\item \textbf{Sección Educativa}: Se incluirá una sección con información
	      científica sobre los impactos de asteroides, explicando los modelos y
	      datos utilizados en la simulación. A medida que ambas simulaciones se
	      completen, se explicaran conceptos científicos y matemáticos relevantes
	      usados.
\end{itemize}
\subsection{Validación de Resultados}
Para asegurar la precisión y confiabilidad de la simulación, se llevará a cabo
un proceso de \textbf{validación} de los resultados obtenidos.
\begin{itemize}
	\item \textbf{Calibración con Modelos Reconocidos}: Se replicarán escenarios de referencia del Earth Impact
	      Effects Program \cite{collins2005} y de los estudios de tsunamis de
	      Ward y Asphaug \cite{ward2000,wuennemann2010}, verificando que los perfiles
	      de cráter, sobrepresión y run-up coincidan dentro de tolerancias del 5
	      al 10\%.
	\item \textbf{Validación Orbital}: Las trayectorias propagadas se contrastarán con efemérides de
	      NASA Mission Visualization y catálogos JPL Horizons, midiendo el error
	      cuadrático medio en posición y velocidad para garantizar consistencia
	      dinámica \cite{nasaEllipticalOrbit}.
	\item \textbf{Casos Históricos}: Se simularán eventos documentados como Tunguska
	      \cite{chyba1993} y Chelyabinsk \cite{popova2013}; los resultados se
	      confrontarán con observables publicados (radios de daño, intensidades de
	      onda de choque, luminosidad) calculando métricas de error relativas.
	\item \textbf{Reproductibilidad y Auditoría}: Se versionarán los conjuntos de
	      parámetros y se publicarán cuadernos de verificación automatizados para
	      que terceros puedan repetir las simulaciones y revisar los balances de
	      energía y momento en cada etapa.
\end{itemize}
\subsection{Optimización y Rendimiento}
Para mantener tiempos de respuesta interactivos se adoptarán estrategias de
\textbf{optimización} en todas las capas de la aplicación.
\begin{itemize}
	\item \textbf{Integración adaptativa}: El integrador orbital y el motor de
	      efectos usarán pasos de tiempo variables guiados por el error local para
	      reducir cómputo en intervalos suaves y concentrarlo en eventos críticos.
	\item \textbf{Computación paralela}: Se delegarán cálculos independientes (p.ej.
	      barridos Monte Carlo y propagación de eyección) a hilos dedicados o Web
	      Workers, manteniendo fluida la interfaz principal.
	\item \textbf{Uso de GPU}: La renderización volumétrica y los mapas de daño se
	      implementarán con WebGL, aprovechando shaders para interpolación y
	      sombreado en tiempo real.
	\item \textbf{Cachés numéricas}: Resultados recurrentes como curvas de
	      Kingery--Bulmash y tablas pi-escaladas se almacenarán en memoria para
	      evitar recomputaciones.
	\item \textbf{Perfilado continuo}: Se instrumentará la aplicación con
	      herramientas de monitoreo (Chrome DevTools, Flamegraphs) para detectar
	      cuellos de botella y establecer presupuestos de fotogramado y memoria.
\end{itemize}

\subsection{Documentación y Presentación}
La entrega incluirá materiales orientados a usuarios finales y revisores
técnicos.
\begin{itemize}
	\item \textbf{Manual de Usuario}: Guía ilustrada que explica configuración de
	      escenarios, interpretación de resultados y límites del modelo.
	\item \textbf{Documento Técnico}: Especificación de modelos físicos,
	      supuestos, validaciones y referencias bibliográficas, enlazando cada
	      fórmula con su implementación.
	\item \textbf{API y Datos}: Descripción de endpoints, contratos de datos y
	      esquemas de exportación (GeoJSON, CSV) para integrar la simulación con
	      otras herramientas.
	\item \textbf{Cuadernos de Verificación}: Jupyter/Markdown que replican casos
	      de prueba y muestran comparativas con estudios de referencia.
	\item \textbf{Elevator Pitch}: Diapositivas con los hallazgos clave,
	      capturas de la interfaz y recomendaciones basadas en los escenarios
	      evaluados.
\end{itemize}

% --- ARQUITECTURA---
\section{Arquitectura}
El sistema propuesto sigue una \textbf{arquitectura cliente-servidor} donde
el \textbf{frontend} interactúa con el \textbf{backend} a través de una API RESTful.
Esta estructura modular permite que los distintos componentes del sistema trabajen
de manera eficiente y escalable. A continuación, se detallan los componentes y
flujo de datos.
\subsection{Descripción General}
La arquitectura se compone de los siguientes módulos principales:
\begin{enumerate}
	\item \textbf{Frontend}: Responsable de la interacción con el usuario,
	      mostrando las simulaciones y permitiendo la entrada de datos (como parámetros del asteroide).
	\item \textbf{Backend}: Se encarga de realizar los cálculos de la simulación
	      (trayectoria, impacto, efectos secundarios) y de integrar los datps de las
	      APIs externas de \textit{NASA} y \textit{USGS}.
\end{enumerate}

\subsection{Componentes del Backend}
El backend estará basado en \textbf{Flask}, un framework ligero de Python, adecuado para
manejar solicitudes HTTP y ejecutar cálculos de simulación. Los principales
componentes del backend son:
\begin{itemize}
	\item \textbf{Tecnologías utilizadas}:
	      \begin{itemize}
		      \item \textbf{Flask}: Se utilizará para desarrollar la API RESTful que
		            procesará las solicitudes del frontend, ejecutará los cálculos de
		            la simulación y devolverá los resultados.
		      \item \textbf{Python}: El lenguaje principal para las simulaciones y
		            procesamiento de datos. Se utilizarán bibliotecas como \textbf{NumPy} y
		            \textbf{SciPy} para cálculos numéricos y \textbf{AstroPy} para cálculos astronómicos.
		      \item \textbf{APIs externas}:
		            \begin{itemize}
			            \item \textbf{NASA NEO API}: Para obtener datos sobre asteroides
			                  cercanos a la Tierra, como órbitas y tamaños.
			            \item \textbf{USGS Earthquake Catalog}: Para modelar los efectos
			                  sísmicos asociados a impactos (terremotos, tsunamis, etc.).
		            \end{itemize}
	      \end{itemize}
	\item \textbf{Flujo de datos}:
	      \begin{itemize}
		      \item El \textbf{frontend} enviará parámetros como tamaño,
		            velocidad y ángulo de entrada del asteroide al backend mediante
		            solicitudes \textbf{POST} a la API.
		      \item El \textbf{backend} procesará estos parámetros, realizará los
		            cálculos de simulación y consultará las APIs externas para obtener
		            datos adicionales. Devolverá los resultados al frontend en formato \textbf{JSON}.
	      \end{itemize}
\end{itemize}

\subsection{Componentes del Frontend}
El \textbf{frontend} se desarrollará utilizando \textbf{React} y \textbf{Vite}
para crear una interfaz de usuario interactiva y eficiente. Los principales
componentes del frontend son:
\begin{itemize}
	\item \textbf{Tecnologías utilizadas}:
	      \begin{itemize}
		      \item \textbf{React}: Framework de JavaScript para construir una
		            interfaz interactiva y gestionada por estado. Será el núcleo del
		            frontend, permitiendo la creación de componentes reutilizables para
		            mostrar la simulación, los resultados y los controles de entrada.
		      \item \textbf{Vite}: Utilizado para la construcción y el desarrollo
		            rápido del frontend. Vite es un \textbf{bundler} de última generación que
		            proporciona tiempos de recarga rápidos y una experiencia de
		            desarrollo eficiente.
		      \item \textbf{Three.js}: Biblioteca para renderizar gráficos 3D en el
		            navegador, utilizada para visualizar la trayectoria del asteroide
		            y los efectos del impacto.
		      \item \textbf{D3.js}: Biblioteca para crear gráficos y visualizaciones
		            de datos en 2D, utilizada para mostrar mapas de daño y otros
		            efectos secundarios.
	      \end{itemize}
	\item \textbf{Comunicación con el Backend}:
	      \begin{itemize}
		      \item El frontend enviará solicitudes \textbf{POST} a la API del backend
		            con los parámetros del asteroide ingresados por el usuario.
	      \end{itemize}
\end{itemize}

\subsection{Flujo de Datos entre Frontend y Backend}
El \textbf{flujo de datos} entre el frontend y el backend se gestionará de la siguiente manera:
\begin{itemize}
	\item El \textbf{frontend} enviará los parámetros seleccionados por el usuario
	      (tamaño, velocidad, ángulo de entrada) al backend a través de una solicitud
	      \textbf{POST} a la API RESTful.
	\item El \textbf{backend} recibirá estos parámetros, los procesará utilizando los
	      modelos científicos para calcular la \textbf{trayectoria} del asteroide y los
	      efectos del impacto. Luego, enviará los resultados de vuelta al
	      frontend en formato \textbf{JSON}.
	\item El \textbf{frontend} actualizará las visualizaciones (en 3D y 2D)
	      en tiempo real a medida que los resultados se reciben del backend.
\end{itemize}

\subsection{Seguridad y Manejo de Errores}
Para garantizar la seguridad y estabilidad del sistema, se implementarán las siguientes medidas:
\begin{itemize}
	\item \textbf{Seguridad}:
	      \begin{itemize}
		      \item \textbf{Validación de entradas}: El backend validará todos los parámetros
		            recibidos del frontend para asegurarse de que sean correctos y estén dentro de
		            los rangos aceptables.
		      \item \textbf{Protección contra inyecciones}: Se implementarán medidas para prevenir ataques de inyección
		            SQL o XSS, asegurando que todas las entradas del usuario sean
		            correctamente sanitizadas.
	      \end{itemize}
	\item \textbf{Manejo de errores}:
	      \begin{itemize}
		      \item \textbf{Manejo de excepciones}: El backend manejará todas las
		            excepciones y errores que puedan ocurrir durante los cálculos o
		            las consultas a las APIs externas, devolviendo mensajes de error
		            claros al frontend.
		      \item \textbf{Notificaciones al usuario}: El frontend mostrará mensajes
		            de error amigables en caso de que ocurra algún problema, como
		            parámetros inválidos o fallos en la comunicación con el backend.
	      \end{itemize}
\end{itemize}


% --- DATOS ABIERTOS NASA ---
\section{Datos abiertos (NASA)}
El desarrollo de este proyecto se basa en la integración de
\textbf{datos abiertos} proporcionados por \textbf{NASA} y \textbf{USGS}, lo
que permite crear simulaciones precisas y realistas sobre los impactos de
asteroides cercanos a la Tierra. Estas fuentes de datos proporcionan
información esencial sobre los \textbf{asteroides} y los
\textbf{efectos geológicos} relacionados con un impacto, y su integración
permite una visualización más detallada y precisa de los riesgos asociados.

\subsection{1. Uso de la \textbf{API de NASA NEO}}
La \textbf{API de NASA Near-Earth Object (NEO)} proporciona acceso a una
vasta base de datos sobre \textbf{asteroides cercanos a la Tierra} (NEOs).
Esta API incluye información crítica como los parámetros orbitales de los
asteroides, su tamaño, velocidad, y las fechas de aproximación cercanas a la
Tierra. Para nuestro proyecto, estos datos se utilizarán para calcular y
simular la \textbf{trayectoria orbital} de los asteroides, lo cual es
fundamental para predecir su impacto en la Tierra.

\begin{itemize}
	\item \textbf{Datos Utilizados}:
	      \begin{itemize}
		      \item \textbf{Parámetros Orbitales}: Semi-eje mayor, excentricidad,
		            inclinación, entre otros.
		      \item \textbf{Tamaño del Asteroide}: Para estimar la energía cinética
		            del impacto.
		      \item \textbf{Fecha y Distancia de Aproximación}: Para simular la
		            trayectoria en el tiempo y predecir el impacto.
	      \end{itemize}
\end{itemize}

La API de NASA proporciona los datos necesarios para crear simulaciones
precisas y determinar la zona de impacto y los posibles efectos secundarios del asteroide.

\subsection{2. Uso de la API de USGS Earthquake Catalog}
La \textbf{API de USGS Earthquake Catalog} ofrece acceso a un catálogo
global de \textbf{datos sísmicos}, incluyendo información sobre la ubicación, magnitud y profundidad de los terremotos. Estos datos serán fundamentales para modelar los efectos sísmicos generados por un impacto de asteroide en la Tierra. Además, se utilizarán los \textbf{modelos de propagación sísmica} para simular las ondas sísmicas generadas por el impacto.

\begin{itemize}
	\item \textbf{Datos Utilizados}:
	      \begin{itemize}
		      \item \textbf{Magnitud de los Terremotos}: Para estimar la fuerza de
		            las ondas sísmicas.
		      \item \textbf{Ubicación y Profundidad}: Para modelar la propagación de
		            las ondas sísmicas en el terreno afectado.
		      \item \textbf{Impacto en Zonas Geográficas}: Para simular los efectos
		            en áreas cercanas al lugar de impacto.
	      \end{itemize}
\end{itemize}

\subsection{3. Integración y Transparencia de Datos}
La integración de estos datos será clave para asegurar que las simulaciones
sean lo más precisas posible. Durante el desarrollo del proyecto, se asegurará
que los datos sean procesados de manera \textbf{transparente}, permitiendo que
los usuarios comprendan cómo se obtienen los datos y cómo afectan los
resultados de las simulaciones. Además, se implementarán mecanismos para
verificar la \textbf{precisión} y \textbf{actualización} de los datos en
tiempo real, utilizando la \textbf{API de NASA NEO} para obtener la
información más actualizada sobre asteroides cercanos y la
\textbf{API de USGS} para obtener los últimos registros sísmicos.

\subsection{4. Aplicación en la Simulación}
Los datos obtenidos a través de estas APIs se integrarán en el \textbf{backend}
del sistema para realizar los cálculos de la trayectoria y los efectos del
impacto. El \textbf{frontend} se encargará de mostrar estos resultados en
tiempo real, permitiendo a los usuarios explorar los diferentes escenarios de
impacto y visualizar cómo los cambios en los parámetros afectan las áreas de
impacto y las consecuencias medioambientales.

% --- ALCANCE ---
\section{Alcance}
Este anteproyecto define el alcance funcional y técnico de la plataforma
\textbf{Meteor Madness}. El desarrollo cubrirá los siguientes aspectos:
\begin{itemize}
	\item \textbf{Simulación física}: Propagación orbital, entrada atmosférica e
	      impacto con generación de mapas de daño para escenarios terrestres y
	      oceánicos utilizando los modelos descritos en la metodología.
	\item \textbf{Interfaz interactiva}: Visualizaciones 3D y 2D en navegador con
	      ajuste de parámetros en tiempo real, indicadores de daño y exportación de
	      resultados.
	\item \textbf{Soporte educativo}: Sección explicativa con fundamentos físicos,
	      referencias y glosario para público no especialista.
	\item \textbf{Infraestructura}: Backend de servicios para cálculo, almacenamiento
	      de escenarios y distribución de datos a clientes web.
\end{itemize}
\textbf{Fuera de alcance} quedan la integración con sistemas de defensa,
alertamiento en tiempo real, modelado climático posterior al impacto y
despliegues móviles nativos; estos requerirían proyectos adicionales.

% --- PLAN DE TRABAJO Y ENTREGABLES ---
\section{Plan de trabajo (48 h) y entregables}
El proyecto se llevará a cabo durante un período de \textbf{48 horas}, donde se
dividirán las actividades en dos días con tareas específicas para lograr un
progreso continuo y bien organizado. A continuación se detalla el plan de
trabajo y los entregables esperados en cada fase.
\subsection{Día 1: Investigación, Diseño y Desarrollo Inicial}
\subsubsection*{Objetivos del Día 1}
\begin{itemize}
	\item Realizar la investigación necesaria sobre los impactos de asteroides y
	      los modelos científicos relevantes.
	\item Definir la estructura general del sistema y el diseño de la interfaz.
	\item Comenzar con el desarrollo de la simulación de la trayectoria del
	      asteroide y los efectos del impacto.
\end{itemize}

\subsubsection*{Actividades}
\begin{enumerate}
	\item \textbf{Investigación y Revisión de Modelos Científicos}
	      \begin{itemize}
		      \item Revisión de modelos de simulación orbital y efectos de impacto
		            (como la formación de cráteres y tsunamis).
		      \item Estudio de las APIs de NASA y USGS para integrar datos sobre
		            asteroides y efectos sísmicos.
	      \end{itemize}
	\item \textbf{Definición de la Arquitectura}
	      \begin{itemize}
		      \item Diseño de la arquitectura cliente-servidor.
		      \item Selección de tecnologías para el frontend (React, Vite,
		            Three.js, D3.js) y backend (Flask, Python).
		      \item Diseño preliminar de la API RESTful para la comunicación entre
		            frontend y backend.
	      \end{itemize}
	\item \textbf{Desarrollo de la Simulación de Trayectoria Orbital}
	      \begin{itemize}
		      \item Implementación de la propagación orbital utilizando los parámetros y
		            ecuaciones de Kepler.
		      \item Validación inicial de la simulación con datos de referencia.
		      \item Configuración de la API de NASA NEO para obtener datos de asteroides cercanos.
	      \end{itemize}
	\item \textbf{Diseño Inicial del Frontend}
	      \begin{itemize}
		      \item Creación de una interfaz básica con React y Vite.
		      \item Implementación de la estructura básica para la visualización
		            de la trayectoria orbital en 3D utilizando Three.js.

	      \end{itemize}
\end{enumerate}

\subsubsection*{Entregables del Día 1}
\begin{itemize}
	\item Prototipo funcional básico con la interfaz de usuario inicial y un
	      esbozo de la visualización en 3D de la trayectoria del asteroide.
	\item Documentación técnica preliminar sobre el diseño y la arquitectura del sistema.
	\item Resultados iniciales de simulación de la trayectoria orbital del asteroide.
	\item Código inicial del backend con la simulación de la trayectoria
	      orbital y la integración con la API de NASA NEO.
\end{itemize}


\subsection{Día 2: Desarrollo Completo, Validación y Documentación}
\subsubsection*{Objetivos del Día 2}
\begin{itemize}
	\item Completar el desarrollo de los efectos del impacto (cráter, tsunamis, terremotos).
	\item Integrar las APIs de NASA y USGS para los datos de asteroides y los efectos sísmicos.
	\item Implementar la interactividad en tiempo real.
	\item Realizar pruebas y optimizar el rendimiento.
\end{itemize}

\subsubsection*{Actividades}
\begin{enumerate}
	\item \textbf{Desarrollo de los Efectos del Impacto}
	      \begin{itemize}
		      \item Implementación de cálculos para la formación de cráteres y el
		            cálculo de la energía cinética del asteroide.
		      \item Simulación de los efectos secundarios del impacto: tsunamis y
		            actividad sísmica, utilizando los datos de USGS.
	      \end{itemize}
	\item \textbf{Integración de Datos de NASA y USGS}
	      \begin{itemize}
		      \item Integración de la API de NASA NEO para obtener datos en
		            tiempo real de asteroides cercanos.
		      \item Integración de la API de USGS Earthquake Catalog para modelar
		            los efectos sísmicos y geológicos del impacto.
	      \end{itemize}
	\item \textbf{Desarrollo de la Interactividad y Visualización}
	      \begin{itemize}
		      \item Creación de sliders interactivos para que los usuarios ajusten
		            parámetros del asteroide (tamaño, velocidad, etc.).
		      \item Implementación de la visualización en 3D de la trayectoria del
		            asteroide y los efectos del impacto.
		      \item Actualización en tiempo real de los mapas 2D con zonas
		            afectadas por tsunamis y terremotos.
	      \end{itemize}
	\item \textbf{Pruebas y Optimización}
	      \begin{itemize}
		      \item Pruebas de usabilidad para asegurar que la herramienta sea
		            fácil de usar y entender.
		      \item Optimización del rendimiento, tanto en el frontend
		            (visualización en 3D) como en los cálculos de simulación.
	      \end{itemize}
\end{enumerate}

\subsubsection*{Entregables del Día 2}
\begin{itemize}
	\item Simulación completa de la trayectoria del asteroide y los efectos del
	      impacto (formación de cráteres, tsunamis, terremotos).
	\item Interfaz interactiva con sliders para modificar parámetros y ver
	      resultados en tiempo real.
	\item Prototipo final de la herramienta web con visualización 3D y mapas 2D interactivos.
	\item Pruebas de usabilidad y optimización de rendimiento.
	\item Documentación final del proyecto, incluyendo instrucciones de uso y detalles técnicos.
\end{itemize}
% --- REFERENCIAS BIBLIOGRAFICAS ---
\begin{thebibliography}{00}

	\bibitem{collins2005}
	G. S. Collins, H. J. Melosh, and R. A. Marcus, ``Earth impact effects program: A web-based computer program for calculating the regional environmental consequences of a meteoroid impact on earth,'' \textit{Meteoritics \& Planetary Science}, vol. 40, no. 6, pp. 817--840, 2005. doi:10.1111/j.1945-5100.2005.tb00157.x

	\bibitem{herrick2006}
	R. R. Herrick, ``Updates regarding the resurfacing of venusian impact craters,'' in \textit{Lunar and Planet. Sci. Conf. XXXVII}, p. Abs. 1588, Lunar and Planetary Institute, Houston, Texas, 2006.

	\bibitem{herrick1997}
	R. R. Herrick, V. L. Sharpton, M. C. Malin, S. N. Lyons, and K. Feely, ``Morphology and Morphometry of Impact Craters,'' in \textit{Venus II: Geology, Geophysics, Atmosphere, and Solar Wind Environment}, S. W. Bougher, D. M. Hunten, and R. J. Phillips, Eds., p. 1015, 1997.

	\bibitem{holsapple1993}
	K. A. Holsapple, ``The scaling of impact processes in planetary sciences,'' \textit{Ann. Rev. Earth Planet. Sci.}, vol. 21, pp. 333--373, 1993.

	\bibitem{mckinnon1985}
	W. B. McKinnon and P. M. Schenk, ``Ejecta blanket scaling on the Moon and Mercury - inferences for projectile populations,'' in \textit{Lunar and Planet. Sci. Conf. Proceedings XVI}, pp. 544--545, Lunar and Planetary Institute, Houston, Texas, 1985.

	\bibitem{wuennemann2010}
	K. Wünnemann, G. S. Collins, and R. Weiss, ``Impact of a cosmic body into earth’s ocean and the generation of large tsunami waves: Insight from numerical modeling,'' \textit{Reviews of Geophysics}, vol. 48, no. 4, 2010. doi:10.1029/2009RG000308

	\bibitem{melosh1989}
	H. J. Melosh, \textit{Impact Cratering: A Geologic Process}. Oxford University Press, 1989.

	\bibitem{housen2011}
	K. R. Housen and K. A. Holsapple, ``Ejecta from impact craters,'' \textit{Icarus}, vol. 211, pp. 856--875, 2011. doi:10.1016/j.icarus.2010.09.017

	\bibitem{mcgetchin1973}
	T. R. McGetchin, M. Settle, and J. W. Head, ``Lunar impact ejection and crater growth,'' \textit{Journal of Geophysical Research}, vol. 78, no. 11, pp. 10847--10863, 1973. doi:10.1029/JB078i023p10847

	\bibitem{chyba1993}
	C. F. Chyba, P. J. Thomas, and K. J. Zahnle, ``The 1908 Tunguska explosion: Atmospheric disruption of a stony asteroid,'' \textit{Nature}, vol. 361, pp. 40--44, 1993. doi:10.1038/361040a0

	\bibitem{popova2013}
	O. P. Popova et al., ``Chelyabinsk Airburst, Damage Assessment, Meteorite Recovery, and Characterization,'' \textit{Science}, vol. 342, no. 6162, pp. 1069--1073, 2013. doi:10.1126/science.1242642

	\bibitem{kingery1984}
	C. N. Kingery and G. Bulmash, ``Airblast Parameters from TNT Spherical Air Burst and Hemispherical Surface Burst,'' Technical Report ARBRL-TR-02555, U.S. Army Ballistic Research Laboratory, Aberdeen Proving Ground, 1984.

	\bibitem{ufc334002}
	UFC 3-340-02, ``Structures to Resist the Effects of Accidental Explosions,'' U.S. Department of Defense, 2008 (Change 2, 2014).

	\bibitem{glasstone1977}
	S. Glasstone and P. J. Dolan, \textit{The Effects of Nuclear Weapons}. U.S. Department of Defense and U.S. Department of Energy, 1977.

	\bibitem{ward2000}
	S. N. Ward and E. Asphaug, ``Asteroid impact tsunami: A probabilistic hazard assessment,'' \textit{Icarus}, vol. 145, pp. 64--78, 2000. doi:10.1006/icar.1999.6336

	\bibitem{nasaEllipticalOrbit}
	NASA Goddard Space Flight Center, ``Elliptical Orbit Design,'' Mission Visualization \url{https://nasa.github.io/mission-viz/RMarkdown/Elliptical_Orbit_Design.html}, accessed Oct. 2024.

\end{thebibliography}
\end{document}
