%% bare_conf.tex
%% V1.4b
%% 2015/08/26
%% by Michael Shell
%% See:
%% http://www.michaelshell.org/
%% for current contact information.
%%
%% This is a skeleton file demonstrating the use of IEEEtran.cls
%% (requires IEEEtran.cls version 1.8b or later) with an IEEE
%% conference paper.
%%
%% Support sites:
%% http://www.michaelshell.org/tex/ieeetran/
%% http://www.ctan.org/pkg/ieeetran
%% and
%% http://www.ieee.org/

%%*************************************************************************
%% Legal Notice:
%% This code is offered as-is without any warranty either expressed or
%% implied; without even the implied warranty of MERCHANTABILITY or
%% FITNESS FOR A PARTICULAR PURPOSE! 
%% User assumes all risk.
%% In no event shall the IEEE or any contributor to this code be liable for
%% any damages or losses, including, but not limited to, incidental,
%% consequential, or any other damages, resulting from the use or misuse
%% of any information contained here.
%%
%% All comments are the opinions of their respective authors and are not
%% necessarily endorsed by the IEEE.
%%
%% This work is distributed under the LaTeX Project Public License (LPPL)
%% ( http://www.latex-project.org/ ) version 1.3, and may be freely used,
%% distributed and modified. A copy of the LPPL, version 1.3, is included
%% in the base LaTeX documentation of all distributions of LaTeX released
%% 2003/12/01 or later.
%% Retain all contribution notices and credits.
%% ** Modified files should be clearly indicated as such, including  **
%% ** renaming them and changing author support contact information. **
%%*************************************************************************


% *** Authors should verify (and, if needed, correct) their LaTeX system  ***
% *** with the testflow diagnostic prior to trusting their LaTeX platform ***
% *** with production work. The IEEE's font choices and paper sizes can   ***
% *** trigger bugs that do not appear when using other class files.       ***                          ***
% The testflow support page is at:
% http://www.michaelshell.org/tex/testflow/



\documentclass[conference]{IEEEtran}
\usepackage{cite}
\usepackage{graphicx}
\usepackage{hyperref}
\usepackage{float}

% Some Computer Society conferences also require the compsoc mode option,
% but others use the standard conference format.
%
% If IEEEtran.cls has not been installed into the LaTeX system files,
% manually specify the path to it like:
% \documentclass[conference]{../sty/IEEEtran}





% Some very useful LaTeX packages include:
% (uncomment the ones you want to load)


% *** MISC UTILITY PACKAGES ***
%
%\usepackage{ifpdf}
% Heiko Oberdiek's ifpdf.sty is very useful if you need conditional
% compilation based on whether the output is pdf or dvi.
% usage:
% \ifpdf
%   % pdf code
% \else
%   % dvi code
% \fi
% The latest version of ifpdf.sty can be obtained from:
% http://www.ctan.org/pkg/ifpdf
% Also, note that IEEEtran.cls V1.7 and later provides a builtin
% \ifCLASSINFOpdf conditional that works the same way.
% When switching from latex to pdflatex and vice-versa, the compiler may
% have to be run twice to clear warning/error messages.






% *** CITATION PACKAGES ***
%
%\usepackage{cite}
% cite.sty was written by Donald Arseneau
% V1.6 and later of IEEEtran pre-defines the format of the cite.sty package
% \cite{} output to follow that of the IEEE. Loading the cite package will
% result in citation numbers being automatically sorted and properly
% "compressed/ranged". e.g., [1], [9], [2], [7], [5], [6] without using
% cite.sty will become [1], [2], [5]--[7], [9] using cite.sty. cite.sty's
% \cite will automatically add leading space, if needed. Use cite.sty's
% noadjust option (cite.sty V3.8 and later) if you want to turn this off
% such as if a citation ever needs to be enclosed in parenthesis.
% cite.sty is already installed on most LaTeX systems. Be sure and use
% version 5.0 (2009-03-20) and later if using hyperref.sty.
% The latest version can be obtained at:
% http://www.ctan.org/pkg/cite
% The documentation is contained in the cite.sty file itself.






% *** GRAPHICS RELATED PACKAGES ***
%
\ifCLASSINFOpdf
  % \usepackage[pdftex]{graphicx}
  % declare the path(s) where your graphic files are
  % \graphicspath{{../pdf/}{../jpeg/}}
  % and their extensions so you won't have to specify these with
  % every instance of \includegraphics
  % \DeclareGraphicsExtensions{.pdf,.jpeg,.png}
\else
  % or other class option (dvipsone, dvipdf, if not using dvips). graphicx
  % will default to the driver specified in the system graphics.cfg if no
  % driver is specified.
  % \usepackage[dvips]{graphicx}
  % declare the path(s) where your graphic files are
  % \graphicspath{{../eps/}}
  % and their extensions so you won't have to specify these with
  % every instance of \includegraphics
  % \DeclareGraphicsExtensions{.eps}
\fi
% graphicx was written by David Carlisle and Sebastian Rahtz. It is
% required if you want graphics, photos, etc. graphicx.sty is already
% installed on most LaTeX systems. The latest version and documentation
% can be obtained at: 
% http://www.ctan.org/pkg/graphicx
% Another good source of documentation is "Using Imported Graphics in
% LaTeX2e" by Keith Reckdahl which can be found at:
% http://www.ctan.org/pkg/epslatex
%
% latex, and pdflatex in dvi mode, support graphics in encapsulated
% postscript (.eps) format. pdflatex in pdf mode supports graphics
% in .pdf, .jpeg, .png and .mps (metapost) formats. Users should ensure
% that all non-photo figures use a vector format (.eps, .pdf, .mps) and
% not a bitmapped formats (.jpeg, .png). The IEEE frowns on bitmapped formats
% which can result in "jaggedy"/blurry rendering of lines and letters as
% well as large increases in file sizes.
%
% You can find documentation about the pdfTeX application at:
% http://www.tug.org/applications/pdftex





% *** MATH PACKAGES ***
%
%\usepackage{amsmath}
% A popular package from the American Mathematical Society that provides
% many useful and powerful commands for dealing with mathematics.
%
% Note that the amsmath package sets \interdisplaylinepenalty to 10000
% thus preventing page breaks from occurring within multiline equations. Use:
%\interdisplaylinepenalty=2500
% after loading amsmath to restore such page breaks as IEEEtran.cls normally
% does. amsmath.sty is already installed on most LaTeX systems. The latest
% version and documentation can be obtained at:
% http://www.ctan.org/pkg/amsmath





% *** SPECIALIZED LIST PACKAGES ***
%
%\usepackage{algorithmic}
% algorithmic.sty was written by Peter Williams and Rogerio Brito.
% This package provides an algorithmic environment fo describing algorithms.
% You can use the algorithmic environment in-text or within a figure
% environment to provide for a floating algorithm. Do NOT use the algorithm
% floating environment provided by algorithm.sty (by the same authors) or
% algorithm2e.sty (by Christophe Fiorio) as the IEEE does not use dedicated
% algorithm float types and packages that provide these will not provide
% correct IEEE style captions. The latest version and documentation of
% algorithmic.sty can be obtained at:
% http://www.ctan.org/pkg/algorithms
% Also of interest may be the (relatively newer and more customizable)
% algorithmicx.sty package by Szasz Janos:
% http://www.ctan.org/pkg/algorithmicx




% *** ALIGNMENT PACKAGES ***
%
%\usepackage{array}
% Frank Mittelbach's and David Carlisle's array.sty patches and improves
% the standard LaTeX2e array and tabular environments to provide better
% appearance and additional user controls. As the default LaTeX2e table
% generation code is lacking to the point of almost being broken with
% respect to the quality of the end results, all users are strongly
% advised to use an enhanced (at the very least that provided by array.sty)
% set of table tools. array.sty is already installed on most systems. The
% latest version and documentation can be obtained at:
% http://www.ctan.org/pkg/array


% IEEEtran contains the IEEEeqnarray family of commands that can be used to
% generate multiline equations as well as matrices, tables, etc., of high
% quality.




% *** SUBFIGURE PACKAGES ***
%\ifCLASSOPTIONcompsoc
%  \usepackage[caption=false,font=normalsize,labelfont=sf,textfont=sf]{subfig}
%\else
%  \usepackage[caption=false,font=footnotesize]{subfig}
%\fi
% subfig.sty, written by Steven Douglas Cochran, is the modern replacement
% for subfigure.sty, the latter of which is no longer maintained and is
% incompatible with some LaTeX packages including fixltx2e. However,
% subfig.sty requires and automatically loads Axel Sommerfeldt's caption.sty
% which will override IEEEtran.cls' handling of captions and this will result
% in non-IEEE style figure/table captions. To prevent this problem, be sure
% and invoke subfig.sty's "caption=false" package option (available since
% subfig.sty version 1.3, 2005/06/28) as this is will preserve IEEEtran.cls
% handling of captions.
% Note that the Computer Society format requires a larger sans serif font
% than the serif footnote size font used in traditional IEEE formatting
% and thus the need to invoke different subfig.sty package options depending
% on whether compsoc mode has been enabled.
%
% The latest version and documentation of subfig.sty can be obtained at:
% http://www.ctan.org/pkg/subfig




% *** FLOAT PACKAGES ***
%
%\usepackage{fixltx2e}
% fixltx2e, the successor to the earlier fix2col.sty, was written by
% Frank Mittelbach and David Carlisle. This package corrects a few problems
% in the LaTeX2e kernel, the most notable of which is that in current
% LaTeX2e releases, the ordering of single and double column floats is not
% guaranteed to be preserved. Thus, an unpatched LaTeX2e can allow a
% single column figure to be placed prior to an earlier double column
% figure.
% Be aware that LaTeX2e kernels dated 2015 and later have fixltx2e.sty's
% corrections already built into the system in which case a warning will
% be issued if an attempt is made to load fixltx2e.sty as it is no longer
% needed.
% The latest version and documentation can be found at:
% http://www.ctan.org/pkg/fixltx2e


%\usepackage{stfloats}
% stfloats.sty was written by Sigitas Tolusis. This package gives LaTeX2e
% the ability to do double column floats at the bottom of the page as well
% as the top. (e.g., "\begin{figure*}[!b]" is not normally possible in
% LaTeX2e). It also provides a command:
%\fnbelowfloat
% to enable the placement of footnotes below bottom floats (the standard
% LaTeX2e kernel puts them above bottom floats). This is an invasive package
% which rewrites many portions of the LaTeX2e float routines. It may not work
% with other packages that modify the LaTeX2e float routines. The latest
% version and documentation can be obtained at:
% http://www.ctan.org/pkg/stfloats
% Do not use the stfloats baselinefloat ability as the IEEE does not allow
% \baselineskip to stretch. Authors submitting work to the IEEE should note
% that the IEEE rarely uses double column equations and that authors should try
% to avoid such use. Do not be tempted to use the cuted.sty or midfloat.sty
% packages (also by Sigitas Tolusis) as the IEEE does not format its papers in
% such ways.
% Do not attempt to use stfloats with fixltx2e as they are incompatible.
% Instead, use Morten Hogholm'a dblfloatfix which combines the features
% of both fixltx2e and stfloats:
%
% \usepackage{dblfloatfix}
% The latest version can be found at:
% http://www.ctan.org/pkg/dblfloatfix




% *** PDF, URL AND HYPERLINK PACKAGES ***
%
%\usepackage{url}
% url.sty was written by Donald Arseneau. It provides better support for
% handling and breaking URLs. url.sty is already installed on most LaTeX
% systems. The latest version and documentation can be obtained at:
% http://www.ctan.org/pkg/url
% Basically, \url{my_url_here}.




% *** Do not adjust lengths that control margins, column widths, etc. ***
% *** Do not use packages that alter fonts (such as pslatex).         ***
% There should be no need to do such things with IEEEtran.cls V1.6 and later.
% (Unless specifically asked to do so by the journal or conference you plan
% to submit to, of course. )


% correct bad hyphenation here
\hyphenation{op-tical net-works semi-conduc-tor}
\usepackage{listings}
\usepackage{xcolor}
\usepackage{algorithm}
\usepackage{algpseudocode}
\usepackage{amsmath}

\begin{document}
%
% paper title
% Titles are generally capitalized except for words such as a, an, and, as,
% at, but, by, for, in, nor, of, on, or, the, to and up, which are usually
% not capitalized unless they are the first or last word of the title.
% Linebreaks \\ can be used within to get better formatting as desired.
% Do not put math or special symbols in the title.
\title{SIAER(Simulador de Impactos de Asteroides y Evaluación de Riesgos): Una
	herramienta web para calcular las consecuencias medioambientales de impactos de
	asteroides en la Tierra}


% author names and affiliations
% use a multiple column layout for up to three different
% affiliations
\author{
	\IEEEauthorblockN{Bendezu Pastrana, Tommy}
	\IEEEauthorblockA{Dept. de Ing. Mecatrónica\\
		Universidad Continental\\
		Huancayo, Perú\\
		tommyhanss007@gmail.com}

	\and

	\IEEEauthorblockN{Meza Millan, Martin}
	\IEEEauthorblockA{Dept. de Ing. Civil\\
		Universidad Continental\\
		Huancayo, Perú\\
		mezamilland@gmail.com}

	\and

	\IEEEauthorblockN{Ildefonso Santos, Steve}
	\IEEEauthorblockA{Dept. de Computer Science\\
		UTEC\\
		Huancayo, Perú\\
		steve.ildefonso@utec.edu.pe}

	\and

	\IEEEauthorblockN{CUALQUIERA}
	\IEEEauthorblockA{Dept. de \\
		NS\\
		, Perú\\
		CORREO}
}

% conference papers do not typically use \thanks and this command
% is locked out in conference mode. If really needed, such as for
% the acknowledgment of grants, issue a \IEEEoverridecommandlockouts
% after \documentclass

% for over three affiliations, or if they all won't fit within the width
% of the page, use this alternative format:
% 
%\author{\IEEEauthorblockN{Michael Shell\IEEEauthorrefmark{1},
%Homer Simpson\IEEEauthorrefmark{2},
%James Kirk\IEEEauthorrefmark{3}, 
%Montgomery Scott\IEEEauthorrefmark{3} and
%Eldon Tyrell\IEEEauthorrefmark{4}}
%\IEEEauthorblockA{\IEEEauthorrefmark{1}School of Electrical and Computer Engineering\\
%Georgia Institute of Technology,
%Atlanta, Georgia 30332--0250\\ Email: see http://www.michaelshell.org/contact.html}
%\IEEEauthorblockA{\IEEEauthorrefmark{2}Twentieth Century Fox, Springfield, USA\\
%Email: homer@thesimpsons.com}
%\IEEEauthorblockA{\IEEEauthorrefmark{3}Starfleet Academy, San Francisco, California 96678-2391\\
%Telephone: (800) 555--1212, Fax: (888) 555--1212}
%\IEEEauthorblockA{\IEEEauthorrefmark{4}Tyrell Inc., 123 Replicant Street, Los Angeles, California 90210--4321}}




% use for special paper notices
%\IEEEspecialpapernotice{(Invited Paper)}




% make the title area
\maketitle

% As a general rule, do not put math, special symbols or citations
% in the abstract
\begin{abstract}

	El NASA Space Apps Challenge 2025 plantea el reto \textit{"Meteor Madness"}, que busca
	desarrollar una herramienta para simular el impacto de asteroides en la Tierra,
	evaluando las consecuencias y ayudando a las autoridades pertinentes a tomar
	medidas preventivas. Este proyecto tiene como objetivo crear una página web
	interactiva que permite a los usuarios ingresar parámetros específicos de un
	asteroide, como su tamaño, velocidad y ángulo de entrada, para calcular y
	visualizar las consecuencias medioambientales. La herramienta integra datos
	reales de la NASA Near-Earth Object (NEO) API y de USGS, lo que proporciona
	una base científica sólida para los cálculos. Los resultados incluyen
	estimaciones sobre la extensión del área afectada, la magnitud de la onda
	expansiva, la generación de tsunamis y otros efectos secundarios. Además, se
	incluye una sección educativa que explica los fundamentos científicos detrás de
	los impactos de asteroides y las medidas de mitigación. Este proyecto no solo
	tiene un impacto directo en la seguridad planetaria, sino que también sirve
	como un punto de partida para futuras investigaciones en la predicción de
	fenómenos astronómicos peligrosos, apoyándose en datos reales y modelos
	científicos.

\end{abstract}

% no keywords




% For peer review papers, you can put extra information on the cover
% page as needed:
% \ifCLASSOPTIONpeerreview
% \begin{center} \bfseries EDICS Category: 3-BBND \end{center}
% \fi
%
% For peerreview papers, this IEEEtran command inserts a page break and
% creates the second title. It will be ignored for other modes.
\IEEEpeerreviewmaketitle

% --- INTRODUCCION ---
\section{Introducción}

El NASA Space Apps Challenge 2025, a través del reto \textit{Meteor Madness},
invita a la comunidad global a desarrollar soluciones innovadoras para entender
y mitigar los riesgos que representan los asteroides cercanos a la Tierra.
Aunque los impactos de asteroides son eventos poco frecuentes, su potencial
destructivo es significativo, con efectos que incluyen tsunamis, terremotos y
alteraciones atmosféricas. A lo largo de la historia, impactos de este tipo han

tenido consecuencias devastadoras, como el evento de Chicxulub, que se cree que
contribuyó a la extinción de los dinosaurios. Esta amenaza subraya la
importancia de contar con herramientas que nos permitan comprender y evaluar
estos riesgos \cite{ward2000,melosh1989}.

Las herramientas interactivas de simulación juegan un papel crucial en la visualización de estos riesgos, ya que permiten explorar diferentes escenarios de impacto y sus consecuencias medioambientales. Estas herramientas no solo ayudan a concienciar al público y a los responsables de la toma de decisiones sobre la magnitud de los impactos, sino que también permiten probar estrategias de mitigación, como la desviación de asteroides. La capacidad de simular cómo los cambios en los parámetros de un asteroide (tamaño, velocidad, trayectoria) afectan el impacto puede proporcionar información valiosa para la planificación de la protección planetaria \cite{collins2005}.

En respuesta a este reto, este anteproyecto propone el desarrollo de un simulador web educativo que permite calcular y visualizar las consecuencias medioambientales de su impacto. Integrando datos reales de la NASA Near-Earth Object (NEO) API y del USGS, el simulador proporcionará estimaciones sobre la extensión del área afectada, la magnitud de la onda expansiva, la formación de cráteres y la generación de tsunamis, todo basado en modelos científicos establecidos. Además, el simulador incluirá una sección educativa que explicará los fundamentos científicos detrás de los impactos de asteroides y las medidas de mitigación \cite{collins2005,wuennemann2010}.


% --- OBJETIVOS ---
\section{Objetivos}

\textbf{Objetivo general}: Desarrollar una herramienta web interactiva que
simule el impacto de asteroides cercanos a la Tierra, permitiendo a los usuarios
visualizar las consecuencias medioambientales y explorar estrategias de
mitigación, utilizando datos reales de la \textit{NASA} y el \textit{USGS}.

\textbf{Objetivos específicos}:
\begin{enumerate}
	\item \textbf{Simular la trayectoria de asteroides cercanos a la Tierra}:
	      \begin{itemize}
		      \item{ Implementar un modelo orbital que permita calcular y
		            visualizar la trayectoria de un asteroide basándose en
		            parámetros Keplerianos.}
	      \end{itemize}

	\item \textbf{Visualizar los efectos medioambientales de un impacto}:
	      \begin{itemize}
		      \item{Calcular, mediante modelos científicos, y mostrar las
		            consecuencias de un impacto de asteroide, tales como la
		            formación de cráteres, la generación de tsunamis, la actividad
		            sísmica asociada, utilizando datos y modelos científicos existentes.}
	      \end{itemize}

	\item \textbf{Integrar datos reales de la \textit{NASA} y el \textit{USGS}}:
	      \begin{itemize}
		      \item{Utilizar la NASA Near-Earth Object (NEO) API para obtener
		            información sobre asteroides cercanos y la USGS Earthquake
		            Catalog para modelar los efectos sísmicos, generando así una
		            simulación realista y precisa.}
	      \end{itemize}

	\item \textbf{Incluir una sección educativa sobre impactos de asteroides}:
	      \begin{itemize}

		      \item{Desarrollar una sección educativa dentro de la herramienta
		            que explique los fundamentos científicos detrás de los
		            impactos de asteroides, las posibles consecuencias
		            medioambientales y las estrategias de mitigación.}
	      \end{itemize}
	\item\textbf{Proveer una interfaz interactiva y accesible para usuarios no expertos}:
	      \begin{itemize}
		      \item{Crear una interfaz amigable que permita a los usuarios
		            modificar parámetros (como el tamaño y la velocidad del
		            asteroide) y ver los resultados en tiempo real, asegurando que la
		            herramienta sea accesible tanto para científicos como para el
		            público general.}


	      \end{itemize}

	\item\textbf{Evaluar y visualizar estrategias de mitigación de impactos}:
	      \begin{itemize}
		      \item{Implementar la simulación de estrategias de mitigación,
		            como la desviación de asteroides, y permitir a los usuarios
		            visualizar cómo estas estrategias afectan la trayectoria del
		            asteroide y los efectos del impacto.}

	      \end{itemize}
\end{enumerate}

% --- METODOLOGIA ---
\section{Metodología}
\subsection{Enfoque del Proyecto}
El proyecto seguirá un enfoque \textbf{iterativo} dividido en varias fases
clave, que se llevarán a cabo de forma paralela cuando sea posible, para
maximizar la eficiencia. Usando la herramienta \textbf{GitHub} tendremos una
evolución paralela en cada aspecto de la app(Frontend, Backend, etc) y
luego se integrarán en un solo producto.

\subsection{Desarrollo de la Simulación}
La simulación se desarrollará en dos partes principales: la \textbf{trayectoria
	de asteroide} y los \textbf{efectos del impacto}.
\begin{itemize}
	\item{\textbf{Simulación de la Trayectoria}: Basándonos en la metodología de
	      diseño de órbitas elípticas de NASA Mission Visualization se definirán
	      los elementos orbitales del asteroide (semi-eje mayor, excentricidad,
	      inclinación, longitud del nodo ascendente y argumento del periastro)
	      para inicializar la simulación. La propagación temporal se realizará
	      resolviendo la ecuación de Kepler para obtener la anomalía excéntrica
	      y, posteriormente, la anomalía verdadera, lo que permitirá calcular la
	      posición y velocidad en un marco centrado en la Tierra. Con estas
	      magnitudes se generará la trayectoria 3D y se estimará el punto de
	      impacto considerando la rotación terrestre, la velocidad de llegada y
	      diferentes ángulos de entrada. \cite{nasaEllipticalOrbit}}
\end{itemize}
\subsection{Interactividad y Visualización}

\subsection{Validación de Resultados}

\subsection{Optimización y Rendimiento}

\subsection{Pruebas y Evaluación}

\subsection{Documentación y Presentación}

% --- ARQUITECTURA---
\section{Arquitectura}

% --- DATOS ABIERTOS NASA ---
\section{Datos abiertos NASA}

% --- ALCANCE ---
\section{Alcance}

% --- PLAN DE TRABAJO Y ENTREGABLES ---
\section{Plan de trabajo (48 h) y entregables}

% --- CONCLUSIONES ---
\section{Conclusiones}



% --- REFERENCIAS BIBLIOGRAFICAS ---
\begin{thebibliography}{00}

	\bibitem{collins2005}
	G. S. Collins, H. J. Melosh, and R. A. Marcus, ``Earth impact effects program: A web-based computer program for calculating the regional environmental consequences of a meteoroid impact on earth,'' \textit{Meteoritics \& Planetary Science}, vol. 40, no. 6, pp. 817--840, 2005. doi:10.1111/j.1945-5100.2005.tb00157.x

	\bibitem{herrick2006}
	R. R. Herrick, ``Updates regarding the resurfacing of venusian impact craters,'' in \textit{Lunar and Planet. Sci. Conf. XXXVII}, p. Abs. 1588, Lunar and Planetary Institute, Houston, Texas, 2006.

	\bibitem{herrick1997}
	R. R. Herrick, V. L. Sharpton, M. C. Malin, S. N. Lyons, and K. Feely, ``Morphology and Morphometry of Impact Craters,'' in \textit{Venus II: Geology, Geophysics, Atmosphere, and Solar Wind Environment}, S. W. Bougher, D. M. Hunten, and R. J. Phillips, Eds., p. 1015, 1997.

	\bibitem{holsapple1993}
	K. A. Holsapple, ``The scaling of impact processes in planetary sciences,'' \textit{Ann. Rev. Earth Planet. Sci.}, vol. 21, pp. 333--373, 1993.

	\bibitem{mckinnon1985}
	W. B. McKinnon and P. M. Schenk, ``Ejecta blanket scaling on the Moon and Mercury - inferences for projectile populations,'' in \textit{Lunar and Planet. Sci. Conf. Proceedings XVI}, pp. 544--545, Lunar and Planetary Institute, Houston, Texas, 1985.

	\bibitem{wuennemann2010}
	K. Wünnemann, G. S. Collins, and R. Weiss, ``Impact of a cosmic body into earth’s ocean and the generation of large tsunami waves: Insight from numerical modeling,'' \textit{Reviews of Geophysics}, vol. 48, no. 4, 2010. doi:10.1029/2009RG000308

	\bibitem{melosh1989}
	H. J. Melosh, \textit{Impact Cratering: A Geologic Process}. Oxford University Press, 1989.

	\bibitem{housen2011}
	K. R. Housen and K. A. Holsapple, ``Ejecta from impact craters,'' \textit{Icarus}, vol. 211, pp. 856--875, 2011. doi:10.1016/j.icarus.2010.09.017

	\bibitem{mcgetchin1973}
	T. R. McGetchin, M. Settle, and J. W. Head, ``Lunar impact ejection and crater growth,'' \textit{Journal of Geophysical Research}, vol. 78, no. 11, pp. 10847--10863, 1973. doi:10.1029/JB078i023p10847

	\bibitem{chyba1993}
	C. F. Chyba, P. J. Thomas, and K. J. Zahnle, ``The 1908 Tunguska explosion: Atmospheric disruption of a stony asteroid,'' \textit{Nature}, vol. 361, pp. 40--44, 1993. doi:10.1038/361040a0

	\bibitem{popova2013}
	O. P. Popova et al., ``Chelyabinsk Airburst, Damage Assessment, Meteorite Recovery, and Characterization,'' \textit{Science}, vol. 342, no. 6162, pp. 1069--1073, 2013. doi:10.1126/science.1242642

	\bibitem{kingery1984}
	C. N. Kingery and G. Bulmash, ``Airblast Parameters from TNT Spherical Air Burst and Hemispherical Surface Burst,'' Technical Report ARBRL-TR-02555, U.S. Army Ballistic Research Laboratory, Aberdeen Proving Ground, 1984.

	\bibitem{ufc334002}
	UFC 3-340-02, ``Structures to Resist the Effects of Accidental Explosions,'' U.S. Department of Defense, 2008 (Change 2, 2014).

	\bibitem{glasstone1977}
	S. Glasstone and P. J. Dolan, \textit{The Effects of Nuclear Weapons}. U.S. Department of Defense and U.S. Department of Energy, 1977.

	\bibitem{ward2000}
	S. N. Ward and E. Asphaug, ``Asteroid impact tsunami: A probabilistic hazard assessment,'' \textit{Icarus}, vol. 145, pp. 64--78, 2000. doi:10.1006/icar.1999.6336

	\bibitem{nasaEllipticalOrbit}
	NASA Goddard Space Flight Center, ``Elliptical Orbit Design,'' Mission Visualization \url{https://nasa.github.io/mission-viz/RMarkdown/Elliptical_Orbit_Design.html}, accessed Oct. 2024.

\end{thebibliography}



\end{document}
