%% bare_conf.tex
%% V1.4b
%% 2015/08/26
%% by Michael Shell
%% See:
%% http://www.michaelshell.org/
%% for current contact information.
%%
%% This is a skeleton file demonstrating the use of IEEEtran.cls
%% (requires IEEEtran.cls version 1.8b or later) with an IEEE
%% conference paper.
%%
%% Support sites:
%% http://www.michaelshell.org/tex/ieeetran/
%% http://www.ctan.org/pkg/ieeetran
%% and
%% http://www.ieee.org/

%%*************************************************************************
%% Legal Notice:
%% This code is offered as-is without any warranty either expressed or
%% implied; without even the implied warranty of MERCHANTABILITY or
%% FITNESS FOR A PARTICULAR PURPOSE! 
%% User assumes all risk.
%% In no event shall the IEEE or any contributor to this code be liable for
%% any damages or losses, including, but not limited to, incidental,
%% consequential, or any other damages, resulting from the use or misuse
%% of any information contained here.
%%
%% All comments are the opinions of their respective authors and are not
%% necessarily endorsed by the IEEE.
%%
%% This work is distributed under the LaTeX Project Public License (LPPL)
%% ( http://www.latex-project.org/ ) version 1.3, and may be freely used,
%% distributed and modified. A copy of the LPPL, version 1.3, is included
%% in the base LaTeX documentation of all distributions of LaTeX released
%% 2003/12/01 or later.
%% Retain all contribution notices and credits.
%% ** Modified files should be clearly indicated as such, including  **
%% ** renaming them and changing author support contact information. **
%%*************************************************************************


% *** Authors should verify (and, if needed, correct) their LaTeX system  ***
% *** with the testflow diagnostic prior to trusting their LaTeX platform ***
% *** with production work. The IEEE's font choices and paper sizes can   ***
% *** trigger bugs that do not appear when using other class files.       ***                          ***
% The testflow support page is at:
% http://www.michaelshell.org/tex/testflow/



\documentclass[conference]{IEEEtran}
\usepackage{cite}
\usepackage{graphicx}
\usepackage{hyperref}
\usepackage{float}

% Some Computer Society conferences also require the compsoc mode option,
% but others use the standard conference format.
%
% If IEEEtran.cls has not been installed into the LaTeX system files,
% manually specify the path to it like:
% \documentclass[conference]{../sty/IEEEtran}





% Some very useful LaTeX packages include:
% (uncomment the ones you want to load)


% *** MISC UTILITY PACKAGES ***
%
%\usepackage{ifpdf}
% Heiko Oberdiek's ifpdf.sty is very useful if you need conditional
% compilation based on whether the output is pdf or dvi.
% usage:
% \ifpdf
%   % pdf code
% \else
%   % dvi code
% \fi
% The latest version of ifpdf.sty can be obtained from:
% http://www.ctan.org/pkg/ifpdf
% Also, note that IEEEtran.cls V1.7 and later provides a builtin
% \ifCLASSINFOpdf conditional that works the same way.
% When switching from latex to pdflatex and vice-versa, the compiler may
% have to be run twice to clear warning/error messages.






% *** CITATION PACKAGES ***
%
%\usepackage{cite}
% cite.sty was written by Donald Arseneau
% V1.6 and later of IEEEtran pre-defines the format of the cite.sty package
% \cite{} output to follow that of the IEEE. Loading the cite package will
% result in citation numbers being automatically sorted and properly
% "compressed/ranged". e.g., [1], [9], [2], [7], [5], [6] without using
% cite.sty will become [1], [2], [5]--[7], [9] using cite.sty. cite.sty's
% \cite will automatically add leading space, if needed. Use cite.sty's
% noadjust option (cite.sty V3.8 and later) if you want to turn this off
% such as if a citation ever needs to be enclosed in parenthesis.
% cite.sty is already installed on most LaTeX systems. Be sure and use
% version 5.0 (2009-03-20) and later if using hyperref.sty.
% The latest version can be obtained at:
% http://www.ctan.org/pkg/cite
% The documentation is contained in the cite.sty file itself.






% *** GRAPHICS RELATED PACKAGES ***
%
\ifCLASSINFOpdf
  % \usepackage[pdftex]{graphicx}
  % declare the path(s) where your graphic files are
  % \graphicspath{{../pdf/}{../jpeg/}}
  % and their extensions so you won't have to specify these with
  % every instance of \includegraphics
  % \DeclareGraphicsExtensions{.pdf,.jpeg,.png}
\else
  % or other class option (dvipsone, dvipdf, if not using dvips). graphicx
  % will default to the driver specified in the system graphics.cfg if no
  % driver is specified.
  % \usepackage[dvips]{graphicx}
  % declare the path(s) where your graphic files are
  % \graphicspath{{../eps/}}
  % and their extensions so you won't have to specify these with
  % every instance of \includegraphics
  % \DeclareGraphicsExtensions{.eps}
\fi
% graphicx was written by David Carlisle and Sebastian Rahtz. It is
% required if you want graphics, photos, etc. graphicx.sty is already
% installed on most LaTeX systems. The latest version and documentation
% can be obtained at: 
% http://www.ctan.org/pkg/graphicx
% Another good source of documentation is "Using Imported Graphics in
% LaTeX2e" by Keith Reckdahl which can be found at:
% http://www.ctan.org/pkg/epslatex
%
% latex, and pdflatex in dvi mode, support graphics in encapsulated
% postscript (.eps) format. pdflatex in pdf mode supports graphics
% in .pdf, .jpeg, .png and .mps (metapost) formats. Users should ensure
% that all non-photo figures use a vector format (.eps, .pdf, .mps) and
% not a bitmapped formats (.jpeg, .png). The IEEE frowns on bitmapped formats
% which can result in "jaggedy"/blurry rendering of lines and letters as
% well as large increases in file sizes.
%
% You can find documentation about the pdfTeX application at:
% http://www.tug.org/applications/pdftex





% *** MATH PACKAGES ***
%
%\usepackage{amsmath}
% A popular package from the American Mathematical Society that provides
% many useful and powerful commands for dealing with mathematics.
%
% Note that the amsmath package sets \interdisplaylinepenalty to 10000
% thus preventing page breaks from occurring within multiline equations. Use:
%\interdisplaylinepenalty=2500
% after loading amsmath to restore such page breaks as IEEEtran.cls normally
% does. amsmath.sty is already installed on most LaTeX systems. The latest
% version and documentation can be obtained at:
% http://www.ctan.org/pkg/amsmath





% *** SPECIALIZED LIST PACKAGES ***
%
%\usepackage{algorithmic}
% algorithmic.sty was written by Peter Williams and Rogerio Brito.
% This package provides an algorithmic environment fo describing algorithms.
% You can use the algorithmic environment in-text or within a figure
% environment to provide for a floating algorithm. Do NOT use the algorithm
% floating environment provided by algorithm.sty (by the same authors) or
% algorithm2e.sty (by Christophe Fiorio) as the IEEE does not use dedicated
% algorithm float types and packages that provide these will not provide
% correct IEEE style captions. The latest version and documentation of
% algorithmic.sty can be obtained at:
% http://www.ctan.org/pkg/algorithms
% Also of interest may be the (relatively newer and more customizable)
% algorithmicx.sty package by Szasz Janos:
% http://www.ctan.org/pkg/algorithmicx




% *** ALIGNMENT PACKAGES ***
%
%\usepackage{array}
% Frank Mittelbach's and David Carlisle's array.sty patches and improves
% the standard LaTeX2e array and tabular environments to provide better
% appearance and additional user controls. As the default LaTeX2e table
% generation code is lacking to the point of almost being broken with
% respect to the quality of the end results, all users are strongly
% advised to use an enhanced (at the very least that provided by array.sty)
% set of table tools. array.sty is already installed on most systems. The
% latest version and documentation can be obtained at:
% http://www.ctan.org/pkg/array


% IEEEtran contains the IEEEeqnarray family of commands that can be used to
% generate multiline equations as well as matrices, tables, etc., of high
% quality.




% *** SUBFIGURE PACKAGES ***
%\ifCLASSOPTIONcompsoc
%  \usepackage[caption=false,font=normalsize,labelfont=sf,textfont=sf]{subfig}
%\else
%  \usepackage[caption=false,font=footnotesize]{subfig}
%\fi
% subfig.sty, written by Steven Douglas Cochran, is the modern replacement
% for subfigure.sty, the latter of which is no longer maintained and is
% incompatible with some LaTeX packages including fixltx2e. However,
% subfig.sty requires and automatically loads Axel Sommerfeldt's caption.sty
% which will override IEEEtran.cls' handling of captions and this will result
% in non-IEEE style figure/table captions. To prevent this problem, be sure
% and invoke subfig.sty's "caption=false" package option (available since
% subfig.sty version 1.3, 2005/06/28) as this is will preserve IEEEtran.cls
% handling of captions.
% Note that the Computer Society format requires a larger sans serif font
% than the serif footnote size font used in traditional IEEE formatting
% and thus the need to invoke different subfig.sty package options depending
% on whether compsoc mode has been enabled.
%
% The latest version and documentation of subfig.sty can be obtained at:
% http://www.ctan.org/pkg/subfig




% *** FLOAT PACKAGES ***
%
%\usepackage{fixltx2e}
% fixltx2e, the successor to the earlier fix2col.sty, was written by
% Frank Mittelbach and David Carlisle. This package corrects a few problems
% in the LaTeX2e kernel, the most notable of which is that in current
% LaTeX2e releases, the ordering of single and double column floats is not
% guaranteed to be preserved. Thus, an unpatched LaTeX2e can allow a
% single column figure to be placed prior to an earlier double column
% figure.
% Be aware that LaTeX2e kernels dated 2015 and later have fixltx2e.sty's
% corrections already built into the system in which case a warning will
% be issued if an attempt is made to load fixltx2e.sty as it is no longer
% needed.
% The latest version and documentation can be found at:
% http://www.ctan.org/pkg/fixltx2e


%\usepackage{stfloats}
% stfloats.sty was written by Sigitas Tolusis. This package gives LaTeX2e
% the ability to do double column floats at the bottom of the page as well
% as the top. (e.g., "\begin{figure*}[!b]" is not normally possible in
% LaTeX2e). It also provides a command:
%\fnbelowfloat
% to enable the placement of footnotes below bottom floats (the standard
% LaTeX2e kernel puts them above bottom floats). This is an invasive package
% which rewrites many portions of the LaTeX2e float routines. It may not work
% with other packages that modify the LaTeX2e float routines. The latest
% version and documentation can be obtained at:
% http://www.ctan.org/pkg/stfloats
% Do not use the stfloats baselinefloat ability as the IEEE does not allow
% \baselineskip to stretch. Authors submitting work to the IEEE should note
% that the IEEE rarely uses double column equations and that authors should try
% to avoid such use. Do not be tempted to use the cuted.sty or midfloat.sty
% packages (also by Sigitas Tolusis) as the IEEE does not format its papers in
% such ways.
% Do not attempt to use stfloats with fixltx2e as they are incompatible.
% Instead, use Morten Hogholm'a dblfloatfix which combines the features
% of both fixltx2e and stfloats:
%
% \usepackage{dblfloatfix}
% The latest version can be found at:
% http://www.ctan.org/pkg/dblfloatfix




% *** PDF, URL AND HYPERLINK PACKAGES ***
%
%\usepackage{url}
% url.sty was written by Donald Arseneau. It provides better support for
% handling and breaking URLs. url.sty is already installed on most LaTeX
% systems. The latest version and documentation can be obtained at:
% http://www.ctan.org/pkg/url
% Basically, \url{my_url_here}.




% *** Do not adjust lengths that control margins, column widths, etc. ***
% *** Do not use packages that alter fonts (such as pslatex).         ***
% There should be no need to do such things with IEEEtran.cls V1.6 and later.
% (Unless specifically asked to do so by the journal or conference you plan
% to submit to, of course. )


% correct bad hyphenation here
\hyphenation{op-tical net-works semi-conduc-tor}
\usepackage{listings}
\usepackage{xcolor}
\usepackage{algorithm}
\usepackage{algpseudocode}
\usepackage{amsmath}

\begin{document}
%
% paper title
% Titles are generally capitalized except for words such as a, an, and, as,
% at, but, by, for, in, nor, of, on, or, the, to and up, which are usually
% not capitalized unless they are the first or last word of the title.
% Linebreaks \\ can be used within to get better formatting as desired.
% Do not put math or special symbols in the title.
\title{MISRA (Meteor Impact Simulator and Risk Assessment): A
	web tool to compute the environmental consequences of meteor impacts on Earth}


% author names and affiliations
% use a multiple column layout for up to three different
% affiliations
\author{
	\IEEEauthorblockN{Bendezu Pastrana, Tommy}
	\IEEEauthorblockA{Dept. of Mechatronics Engineering\\
		Universidad Continental\\
		Huancayo, Peru\\
		\href{mailto:tommyhanss007@gmail.com}{tommyhanss007@gmail.com}}

	\and

	\IEEEauthorblockN{Ildefonso Santos, Steve}
	\IEEEauthorblockA{Dept. of Computer Science\\
		UTEC\\
		Huancayo, Peru\\
		\href{mailto:steve.ildefonso@utec.edu.pe}{steve.ildefonso@utec.edu.pe}}
}

% conference papers do not typically use \thanks and this command
% is locked out in conference mode. If really needed, such as for
% the acknowledgment of grants, issue a \IEEEoverridecommandlockouts
% after \documentclass

% for over three affiliations, or if they all won't fit within the width
% of the page, use this alternative format:
% 
%\author{\IEEEauthorblockN{Michael Shell\IEEEauthorrefmark{1},
%Homer Simpson\IEEEauthorrefmark{2},
%James Kirk\IEEEauthorrefmark{3}, 
%Montgomery Scott\IEEEauthorrefmark{3} and
%Eldon Tyrell\IEEEauthorrefmark{4}}
%\IEEEauthorblockA{\IEEEauthorrefmark{1}School of Electrical and Computer Engineering\\
%Georgia Institute of Technology,
%Atlanta, Georgia 30332--0250\\ Email: see http://www.michaelshell.org/contact.html}
%\IEEEauthorblockA{\IEEEauthorrefmark{2}Twentieth Century Fox, Springfield, USA\\
%Email: homer@thesimpsons.com}
%\IEEEauthorblockA{\IEEEauthorrefmark{3}Starfleet Academy, San Francisco, California 96678-2391\\
%Telephone: (800) 555--1212, Fax: (888) 555--1212}
%\IEEEauthorblockA{\IEEEauthorrefmark{4}Tyrell Inc., 123 Replicant Street, Los Angeles, California 90210--4321}}




% use for special paper notices
%\IEEEspecialpapernotice{(Invited Paper)}




% make the title area
\maketitle

% As a general rule, do not put math, special symbols or citations
% in the abstract
\begin{abstract}

	The NASA Space Apps Challenge 2025 introduces the \textit{"Meteor Madness"}
	challenge, which seeks to develop a tool capable of simulating meteor impacts on
	Earth, assessing their consequences, and helping the relevant authorities implement
	preventive measures. This project aims to build an interactive web application that
	enables users to enter specific meteor parameters, such as size, velocity, and
	entry angle, to compute and visualize environmental consequences. The tool
	integrates real data from the NASA Near-Earth Object (NEO) API and the USGS,
	providing a solid scientific foundation for the calculations. The results include
	estimates of the extent of the affected area, the magnitude of the blast wave,
	tsunami generation, and other secondary effects. The application also features an
	educational section that explains the scientific principles behind meteor impacts
	and mitigation measures. Beyond its direct contribution to planetary safety, the
	project serves as a starting point for future research on predicting hazardous
	astronomical phenomena, leveraging real data and scientific models.

\end{abstract}

% no keywords




% For peer review papers, you can put extra information on the cover
% page as needed:
% \ifCLASSOPTIONpeerreview
% \begin{center} \bfseries EDICS Category: 3-BBND \end{center}
% \fi
%
% For peerreview papers, this IEEEtran command inserts a page break and
% creates the second title. It will be ignored for other modes.
\IEEEpeerreviewmaketitle

% --- INTRODUCTION ---
\section{Introduction}

The NASA Space Apps Challenge 2025, through the \textit{Meteor Madness} prompt,
invites the global community to develop innovative solutions that help
understand and mitigate the risks posed by near-Earth meteors. Although
meteor impacts are rare events, their destructive potential is significant,
with effects that include tsunamis, earthquakes, and atmospheric disturbances.
Throughout history, impacts of this kind have produced devastating
consequences, such as the Chicxulub event, which is believed to have
contributed to the extinction of the dinosaurs. This threat underscores the
importance of tools that allow us to understand and evaluate these risks
\cite{ward2000,melosh1989}.

Interactive simulation tools play a crucial role in visualizing these risks,
because they allow users to explore different impact scenarios and the
associated environmental consequences. These tools not only raise awareness
among the public and decision-makers regarding the magnitude of impacts, but
also make it possible to test mitigation strategies such as meteor
deflection. The ability to simulate how changes in a meteor's parameters
(size, velocity, trajectory) affect the outcome can provide valuable
information for planetary protection planning \cite{collins2005}.

In response to this challenge, this preliminary project proposes the
development of an educational web simulator that calculates and visualizes the
environmental consequences of an impact. By integrating real data from the NASA
Near-Earth Object (NEO) API and the USGS, the simulator will estimate the
extent of the affected area, the magnitude of the blast wave, crater formation,
and tsunami generation, all grounded in established scientific models. The
simulator will also include an educational section that explains the scientific
principles behind meteor impacts and mitigation measures
\cite{collins2005,wuennemann2010}.


% --- OBJECTIVES ---
\section{Objectives}

\textbf{General objective}: Develop an interactive web tool that simulates the
impact of near-Earth meteors, allowing users to visualize environmental
consequences and explore mitigation strategies using real data from
\textit{NASA} and the \textit{USGS}.

\textbf{Specific objectives}:
\begin{enumerate}
	\item \textbf{Simulate the trajectory of near-Earth meteors}:
	      \begin{itemize}
		      \item{ Implement an orbital model that calculates and visualizes an
		            meteor's trajectory based on Keplerian parameters.}
	      \end{itemize}

	\item \textbf{Visualize the environmental effects of an impact}:
	      \begin{itemize}
		      \item{ Compute, using scientific models, and display the consequences
		            of a meteor impact, such as crater formation, tsunami
		            generation, and associated seismic activity, relying on existing
		            scientific data and models.}
	      \end{itemize}

	\item \textbf{Integrate real data from \textit{NASA} and the \textit{USGS}}:
	      \begin{itemize}
		      \item{ Use the NASA Near-Earth Object (NEO) API to retrieve
		            information about nearby meteors and the USGS Earthquake
		            Catalog to model seismic effects, producing a realistic and
		            accurate simulation.}
	      \end{itemize}

	\item \textbf{Include an educational section on meteor impacts}:
	      \begin{itemize}
		      \item{ Develop an educational section within the tool that explains the
		            scientific foundations behind meteor impacts, the potential
		            environmental consequences, and mitigation strategies.}
	      \end{itemize}

	\item \textbf{Provide an interactive and accessible interface for non-expert users}:
	      \begin{itemize}
		      \item{ Create a user-friendly interface that allows users to adjust
		            parameters (such as meteor size and velocity) and see results
		            in real time, ensuring the tool is accessible to both scientists
		            and the general public.}
	      \end{itemize}

	\item \textbf{Evaluate and visualize impact mitigation strategies}:
	      \begin{itemize}
		      \item{ Implement the simulation of mitigation strategies, such as
		            meteor deflection, and allow users to visualize how these
		            strategies affect the meteor's trajectory and the resulting
		            impact effects.}
	      \end{itemize}
\end{enumerate}

% --- METHODOLOGY ---
\section{Methodology}
\subsection{Project Approach}
The project will follow an \textbf{iterative} approach divided into several key
phases that will be executed in parallel whenever possible to maximize
efficiency. Using \textbf{GitHub}, we will evolve each aspect of the app (frontend,
backend, etc.) in parallel and then integrate them into a single product.

\subsection{Simulation Development}
The simulation will be developed in two main parts: the \textbf{meteor
	trajectory} and the \textbf{impact effects}.
\begin{itemize}
	\item \textbf{Trajectory Simulation}: Building on the elliptical orbit design
	      methodology from NASA Mission Visualization, we will define the meteor's
	      orbital elements (semi-major axis, eccentricity, inclination, right
	      ascension of the ascending node, and argument of periapsis) to initialize
	      the simulation. Temporal propagation will be performed by solving Kepler's
	      equation,
	      \begin{align}
		      M(t) & = n\,(t - t_0) = E - e \sin E, \label{eq:kepler}
	      \end{align}
	      to obtain the eccentric anomaly $E$. The true anomaly $\nu$ will be
	      computed via
	      \begin{align}
		      \tan\frac{\nu}{2} & = \sqrt{\frac{1 + e}{1 - e}}\,\tan\frac{E}{2},
	      \end{align}
	      while the distance to the focus follows
	      \begin{align}
		      r & = a\,(1 - e \cos E).
	      \end{align}
	      With these magnitudes we will evaluate the position vector in an
	      Earth-centered frame by applying Euler rotations over the orbital plane,
	      \begin{align}
		      \mathbf{r}_{\text{ECI}} & =
		      R_3(-\Omega)\,R_1(-i)\,R_3(-\omega)
		      \begin{bmatrix}
			      r \cos \nu \\
			      r \sin \nu \\
			      0
		      \end{bmatrix},
	      \end{align}
	      which will allow us to compute position and velocity in an Earth-centered
	      frame. These quantities will generate the three-dimensional trajectory and
	      estimate the impact point by accounting for Earth's rotation, the arrival
	      velocity, and different entry angles \cite{nasaEllipticalOrbit}.
	\item \textbf{Impact Effects Simulation}: We will follow the workflow
	      established in the Earth Impact Effects Program \cite{collins2005},
	      complemented with blast engineering references. The initial projectile
	      properties are defined as
	      \begin{align}
		      m   & = \tfrac{\pi}{6}\,\rho_i D_i^3,          &
		      R_i & = \tfrac{D_i}{2}, \label{eq:impact-mass}
	      \end{align}
	      and the atmospheric evolution will be integrated by solving the
	      deceleration and ablation equations \cite{collins2005},
	      \begin{align}
		      \frac{dv}{dt} & = -\frac{C_D A \rho_{\text{atm}}(h)}{2 m}\,v^2 - g\sin\gamma, \label{eq:drag} \\
		      \frac{dm}{dt} & = -\frac{C_h A \rho_{\text{atm}}(h)}{2 Q^*}\,v^3, \label{eq:ablation}
	      \end{align}
	      evaluating breakup when the dynamic pressure exceeds the material
	      strength according to
	      \begin{align}
		      q(h) & = \tfrac{1}{2}\,\rho_{\text{atm}}(h)\,v^2 \ge \sigma_c. \label{eq:breakup}
	      \end{align}
	      If the body reaches the ground, we compute the kinetic energy and its
	      equivalent in kilotons of TNT
	      \begin{align}
		      E_k           & = \tfrac{1}{2} m v_{\text{impact}}^2,                           &
		      W_{\text{kt}} & = \frac{E_k}{4.184 \times 10^{12}\ \text{J}}, \label{eq:energy}
	      \end{align}
	      distributing the energy among seismic, thermal, and blast fractions
	      following the efficiencies in \cite{collins2005,glasstone1977}. For crater
	      sizing we apply Holsapple's dimensionless pi-scaling laws adopted by
	      \cite{collins2005}:
	      \begin{align}
		      \pi_2 & = \frac{g R_i}{v_{\text{impact}}^2},                           &
		      \pi_3 & = \frac{Y_t}{\rho_t v_{\text{impact}}^2}, \label{eq:pi-groups}
	      \end{align}
	      with $Y_t$ representing target strength. The transient crater radius is
	      obtained by taking the maximum of the gravity and strength regimes
	      \cite{collins2005,holsapple1993},
	      \begin{align}
		      R_t^{(g)} & = K_g R_i \Bigl(\frac{\rho_i}{\rho_t}\Bigr)^{1/3}\pi_2^{-\mu_g}, \label{eq:rt-gravity}  \\
		      R_t^{(s)} & = K_s R_i \Bigl(\frac{\rho_i}{\rho_t}\Bigr)^{1/3}\pi_3^{-\mu_s}, \label{eq:rt-strength} \\
		      R_t       & = \max\bigl(R_t^{(g)}, R_t^{(s)}\bigr), \label{eq:rt-final}
	      \end{align}
	      with coefficients $K_g = 1.161$, $\mu_g = 0.22$, $K_s = 1.03$, and
	      $\mu_s = 0.275$ for rocky targets. The final diameter distinguishes
	      between simple and complex craters \cite{collins2005,herrick1997},
	      \begin{align}
		      D_f & =
		      \begin{cases}
			      1.25\,D_t,                              & D_t \le D_{\text{sc}}, \\
			      1.17\,D_t^{1.13} D_{\text{sc}}^{-0.13}, & D_t > D_{\text{sc}},
		      \end{cases}\label{eq:final-crater}
	      \end{align}
	      where $D_t = 2 R_t$ and $D_{\text{sc}}$ is the local simple-to-complex
	      transition threshold. Ballistic ejecta will be modeled with the radial
	      decay laws from the program \cite{collins2005,herrick2006},
	      \begin{align}
		      h_e(r) & = 0.14 R_t \Bigl(\frac{R_t}{r}\Bigr)^3, \label{eq:ejecta-thickness} \\
		      v_e(r) & = \sqrt{2 g R_t}\Bigl(\frac{R_t}{r}\Bigr)^{3/2},
	      \end{align}
	      which provide mantle thickness estimates and deposition velocities. For
	      atmospheric effects we adopt the explosive coupling efficiency $\eta_b$
	      proposed in \cite{collins2005} and use the Kingery--Bulmash scaling
	      \cite{ufc334002} on the equivalent charge $W_{\text{eq}} = \eta_b
		      W_{\text{kt}}$,
	      \begin{align}
		      Z           & = \frac{R}{W_{\text{eq}}^{1/3}},                  \\
		      \Delta P(Z) & = \exp\Biggl(\sum_{i=0}^{5} a_i (\ln Z)^i\Biggr), \\
		      I(Z)        & = \exp\Biggl(\sum_{i=0}^{5} b_i (\ln Z)^i\Biggr)
	      \end{align}
	      assessing the dynamic structural load through
	      \begin{align}
		      q_{\text{din}} & = \Delta P(Z) + \tfrac{1}{2} \rho_{\text{air}} U_s^2(Z),
	      \end{align}
	      where $U_s(Z)$ comes from the Rankine--Hugoniot relation
	      \cite{ufc334002}. The thermal component uses the luminous efficiency
	      $\eta_L$ tabulated in \cite{collins2005},
	      \begin{align}
		      H(R) & = \eta_L\,\frac{E_k}{4 \pi R^2 \tau},
	      \end{align}
	      integrating the exposure over duration $\tau$ to derive stage-specific
	      doses. For ocean impacts we retain the cavity and tsunami modeling by Ward
	      and Asphaug \cite{ward2000},
	      \begin{align}
		      r_c     & = \Biggl(\frac{3 E_k}{2 \pi \rho_w g}\Biggr)^{1/3}, &
		      \eta(r) & = \eta_0 \frac{r_c}{r}\,
		      \exp\Bigl(-\frac{r - r_c}{L_d}\Bigr),
	      \end{align}
	      propagating the coastal \textit{run-up} through Green's law,
	      \begin{align}
		      R_{\text{run-up}} & = \eta(r_s)
		      \Biggl(\frac{h_s}{h(r_s)}\Biggr)^{1/4}.
	      \end{align}
	      This chain of models covers atmospheric burst, land, and ocean impact
	      scenarios and delivers thermal, structural, and inundation damage maps
	      consistent with the referenced sources.
\end{itemize}

\subsection{Interactivity and Visualization}
The interface will be designed to let users input, adjust, and visualize the
meteor parameters and simulation results in real time.
\begin{itemize}
	\item \textbf{Parameter Input}: Users will be able to modify parameters such as
	      \textbf{size}, \textbf{velocity}, and \textbf{entry angle} through
	      \textbf{sliders} and \textbf{text fields}.
	\item \textbf{3D Visualization}: \textbf{Three.js} will render the
	      \textbf{meteor trajectory} in an interactive 3D environment, allowing
	      users to rotate, zoom in, and zoom out.
	\item \textbf{2D Visualization}: \textbf{D3.js} will produce charts and maps
	      that represent the \textbf{impact effects}, such as the extent of the
	      affected area, blast-wave magnitude, and tsunami generation.
	\item \textbf{Real-time Updates}: Simulation results will update dynamically as
	      users modify the parameters, providing immediate feedback.
	\item \textbf{Educational Section}: A section with scientific information on
	      meteor impacts will explain the models and data used in the simulation.
	      As both simulations are completed, we will describe the relevant
	      scientific and mathematical concepts employed.
\end{itemize}
\subsection{Result Validation}
To ensure the accuracy and reliability of the simulation, we will conduct a
\textbf{validation} process on the results obtained.
\begin{itemize}
	\item \textbf{Calibration with Recognized Models}: We will reproduce reference
	      scenarios from the Earth Impact Effects Program \cite{collins2005} and
	      tsunami studies by Ward and Asphaug \cite{ward2000,wuennemann2010},
	      verifying that crater, overpressure, and run-up profiles agree within
	      5--10\% tolerances.
	\item \textbf{Orbital Validation}: Propagated trajectories will be contrasted
	      with ephemerides from NASA Mission Visualization and JPL Horizons
	      catalogs, measuring the root-mean-square error in position and velocity to
	      guarantee dynamic consistency \cite{nasaEllipticalOrbit}.
	\item \textbf{Historical Cases}: Documented events such as Tunguska
	      \cite{chyba1993} and Chelyabinsk \cite{popova2013} will be simulated;
	      the results will be compared with published observations (damage radii,
	      shock-wave intensities, luminosity) calculating relative error metrics.
	\item \textbf{Reproducibility and Auditability}: Parameter sets will be
	      versioned and automated verification notebooks will be published so that
	      third parties can repeat the simulations and review energy and momentum
	      balances at each stage.
\end{itemize}
\subsection{Optimization and Performance}
To maintain interactive response times we will adopt \textbf{optimization}
strategies across every layer of the application.
\begin{itemize}
	\item \textbf{Adaptive Integration}: The orbital integrator and the effects
	      engine will use variable time steps driven by local error to reduce
	      computation in smooth intervals and focus effort on critical events.
	\item \textbf{Parallel Computing}: Independent calculations (e.g., Monte Carlo
	      sweeps and ejecta propagation) will be delegated to dedicated threads or
	      Web Workers, keeping the main interface responsive.
	\item \textbf{GPU Utilization}: Volumetric rendering and damage maps will be
	      implemented with WebGL, leveraging shaders for real-time interpolation and
	      shading.
	\item \textbf{Numerical Caching}: Recurrent results such as Kingery--Bulmash
	      curves and pi-scaled tables will be stored in memory to avoid
	      recomputation.
	\item \textbf{Continuous Profiling}: The application will be instrumented with
	      monitoring tools (Chrome DevTools, flame graphs) to detect bottlenecks and
	      establish frame-rate and memory budgets.
\end{itemize}

\subsection{Documentation and Presentation}
The delivery will include materials aimed at end users and technical reviewers.
\begin{itemize}
	\item \textbf{User Manual}: An illustrated guide explaining scenario setup,
	      result interpretation, and model limitations.
	\item \textbf{Technical Document}: A specification of physical models,
	      assumptions, validations, and references, linking each equation to its
	      implementation.
	\item \textbf{API and Data}: A description of endpoints, data contracts, and
	      export schemes (GeoJSON, CSV) to integrate the simulation with other
	      tools.
	\item \textbf{Verification Notebooks}: Jupyter/Markdown artifacts that
	      replicate test cases and present comparisons with reference studies.
	\item \textbf{Elevator Pitch}: Slides summarizing key findings, interface
	      screenshots, and recommendations based on the evaluated scenarios.
\end{itemize}

% --- ARCHITECTURE ---
\section{Architecture}
The proposed system follows a \textbf{client-server architecture} where the
\textbf{frontend} interacts with the \textbf{backend} through a RESTful API. This
modular structure enables the different components of the system to operate
efficiently and scale. The following subsections describe the components and
data flow in detail.
\subsection{Overview}
The architecture consists of the following main modules:
\begin{enumerate}
	\item \textbf{Frontend}: Responsible for user interaction, displaying the
	      simulations, and enabling data input (such as meteor parameters).
	\item \textbf{Backend}: Handles the simulation computations (trajectory,
	      impact, secondary effects) and integrates external data from
	      \textit{NASA} and the \textit{USGS} APIs.
\end{enumerate}

\subsection{Backend Components}
The backend will be built with \textbf{Flask}, a lightweight Python framework
well suited for handling HTTP requests and running simulation calculations. The
main backend components are:
\begin{itemize}
	\item \textbf{Technologies used}:
	      \begin{itemize}
		      \item \textbf{Flask}: Used to develop the RESTful API that processes
		            frontend requests, executes the simulation calculations, and
		            returns the results.
		      \item \textbf{Python}: The primary language for simulations and data
		            processing. Libraries such as \textbf{NumPy} and \textbf{SciPy}
		            will support numerical work, while \textbf{AstroPy} will cover
		            astronomical computations.
		      \item \textbf{External APIs}:
		            \begin{itemize}
			            \item \textbf{NASA NEO API}: Provides data about near-Earth
			                  meteors, including orbits and sizes.
			            \item \textbf{USGS Earthquake Catalog}: Supplies seismic data
			                  associated with impacts (earthquakes, tsunamis, etc.).
		            \end{itemize}
	      \end{itemize}
	\item \textbf{Data flow}:
	      \begin{itemize}
		      \item The \textbf{frontend} will send parameters such as size,
		            velocity, and entry angle to the backend through \textbf{POST}
		            requests to the API.
		      \item The \textbf{backend} will process these parameters, perform the
		            simulation calculations, and query the external APIs for
		            additional data, returning the results to the frontend in
		            \textbf{JSON} format.
	      \end{itemize}
\end{itemize}

\subsection{Frontend Components}
The \textbf{frontend} will be developed with \textbf{React} and \textbf{Vite} to
deliver an interactive and efficient user interface. The main frontend
components are:
\begin{itemize}
	\item \textbf{Technologies used}:
	      \begin{itemize}
		      \item \textbf{React}: JavaScript framework for building an interactive,
		            state-managed interface. It will serve as the core of the frontend
		            and enable reusable components to display the simulation, results,
		            and input controls.
		      \item \textbf{Vite}: Provides fast development and build workflows.
		            Vite is a next-generation \textbf{bundler} that offers rapid
		            reload times and an efficient developer experience.
		      \item \textbf{Three.js}: Library used to render 3D graphics in the
		            browser, enabling visualization of the meteor trajectory and
		            impact effects.
		      \item \textbf{D3.js}: Library for creating 2D data visualizations and
		            maps to portray damage footprints and other secondary effects.
	      \end{itemize}
	\item \textbf{Communication with the Backend}:
	      \begin{itemize}
		      \item The frontend will send \textbf{POST} requests to the backend API
		            containing the meteor parameters entered by the user.
	      \end{itemize}
\end{itemize}

\subsection{Data Flow Between Frontend and Backend}
The \textbf{data flow} between the frontend and backend will be managed as
follows:
\begin{itemize}
	\item The \textbf{frontend} will send the parameters selected by the user
	      (size, velocity, entry angle) to the backend through a \textbf{POST}
	      request to the RESTful API.
	\item The \textbf{backend} will receive these parameters, process them using the
	      scientific models to compute the meteor \textbf{trajectory} and impact
	      effects, and then return the results to the frontend in \textbf{JSON}
	      format.
	\item The \textbf{frontend} will update the visualizations (3D and 2D) in real
	      time as results are received from the backend.
\end{itemize}

\subsection{Security and Error Handling}
To guarantee system security and stability, the following measures will be
implemented:
\begin{itemize}
	\item \textbf{Security}:
	      \begin{itemize}
		      \item \textbf{Input validation}: The backend will validate all
		            parameters received from the frontend to ensure they are correct
		            and within acceptable ranges.
		      \item \textbf{Injection protection}: Measures will be implemented to
		            prevent SQL or XSS injection attacks, ensuring that all user
		            inputs are properly sanitized.
	      \end{itemize}
	\item \textbf{Error handling}:
	      \begin{itemize}
		      \item \textbf{Exception management}: The backend will handle all
		            exceptions and errors that may occur during calculations or when
		            querying external APIs, returning clear error messages to the
		            frontend.
		      \item \textbf{User notifications}: The frontend will display
		            user-friendly error messages if problems arise, such as invalid
		            parameters or communication failures with the backend.
	      \end{itemize}
\end{itemize}


% --- NASA OPEN DATA ---
\section{Open Data (NASA)}
The development of this project relies on integrating \textbf{open data}
provided by \textbf{NASA} and the \textbf{USGS}, which enables precise and
realistic simulations of near-Earth meteor impacts. These data sources offer
essential information about \textbf{meteors} and the \textbf{geological
	effects} associated with an impact, making it possible to visualize the related
risks in greater detail.

\subsection{1. Use of the \textbf{NASA NEO API}}
The \textbf{NASA Near-Earth Object (NEO) API} provides access to a vast
database of \textbf{near-Earth meteors} (NEOs). This API includes critical
information such as orbital parameters, size, velocity, and close approach dates
to Earth. For our project, these data will be used to calculate and simulate the
meteors' \textbf{orbital trajectories}, which is fundamental for predicting an
impact on Earth.

\begin{itemize}
	\item \textbf{Data used}:
	      \begin{itemize}
		      \item \textbf{Orbital parameters}: Semi-major axis, eccentricity,
		            inclination, and related elements.
		      \item \textbf{Meteor size}: Needed to estimate the kinetic energy of
		            the impact.
		      \item \textbf{Approach date and distance}: Supports time-dependent
		            trajectory simulations and impact prediction.
	      \end{itemize}
\end{itemize}

The NASA API provides the data required to create accurate simulations and to
determine the impact zone and potential secondary effects of the meteor.

\subsection{2. Use of the USGS Earthquake Catalog API}
The \textbf{USGS Earthquake Catalog API} provides access to a global catalog of
\textbf{seismic data}, including information about earthquake location,
magnitude, and depth. These data are essential for modeling the seismic effects
generated by a meteor impact on Earth. We will also employ \textbf{seismic
	propagation models} to simulate the waves caused by the impact.

\begin{itemize}
	\item \textbf{Data used}:
	      \begin{itemize}
		      \item \textbf{Earthquake magnitude}: To estimate the strength of the
		            seismic waves.
		      \item \textbf{Location and depth}: To model how seismic waves propagate
		            through the affected terrain.
		      \item \textbf{Impact on geographic areas}: To simulate effects in
		            regions near the impact site.
	      \end{itemize}
\end{itemize}

\subsection{3. Data Integration and Transparency}
Integrating these datasets is key to ensuring the simulations are as accurate as
possible. Throughout the project we will guarantee that the data are processed
in a \textbf{transparent} manner, allowing users to understand where the data
come from and how they influence the simulation outcomes. We will also
implement mechanisms to verify the \textbf{accuracy} and \textbf{freshness} of
the data in real time, using the \textbf{NASA NEO API} for the latest near-Earth
meteor information and the \textbf{USGS API} for recent seismic records.

\subsection{4. Application within the Simulation}
The data obtained through these APIs will be integrated into the system
\textbf{backend} to perform the trajectory and impact-effect calculations. The
\textbf{frontend} will display these results in real time, enabling users to
explore different impact scenarios and to see how parameter changes affect the
impact areas and environmental consequences.

% --- SCOPE ---
\section{Scope}
This preliminary project defines the functional and technical scope of the
\textbf{MISRA} platform. Development will cover the following aspects:
\begin{itemize}
	\item \textbf{Physical simulation}: Orbital propagation, atmospheric entry, and
	      impact modeling with damage maps for terrestrial and ocean scenarios
	      using the models described in the methodology.
	\item \textbf{Interactive interface}: 3D and 2D browser visualizations with
	      real-time parameter adjustments, damage indicators, and data export.
	\item \textbf{Educational support}: An explanatory section with physical
	      foundations, references, and a glossary for non-specialist audiences.
	\item \textbf{Infrastructure}: A service backend for computation, scenario
	      storage, and data distribution to web clients.
\end{itemize}
\textbf{Out of scope} are integration with defense systems, real-time alerting,
post-impact climate modeling, and native mobile deployments; these would require
additional projects.

% --- WORK PLAN AND DELIVERABLES ---
\section{Work Plan (48 h) and Deliverables}
The project will be executed over a \textbf{48-hour} period, splitting the
activities into two days with specific tasks to maintain steady, well-organized
progress. The following sections describe the work plan and the expected
deliverables for each phase.
\subsection{Day 1: Research, Design, and Initial Development}
\subsubsection*{Day 1 Objectives}
\begin{itemize}
	\item Conduct the necessary research on meteor impacts and the relevant
	      scientific models.
	\item Define the overall system structure and interface design.
	\item Begin developing the meteor trajectory simulation and the impact
	      effects module.
\end{itemize}

\subsubsection*{Activities}
\begin{enumerate}
	\item \textbf{Research and Review of Scientific Models}
	      \begin{itemize}
		      \item Review orbital simulation models and impact effects (such as crater
		            formation and tsunamis).
		      \item Study the NASA and USGS APIs to integrate meteor and seismic
		            data.
	      \end{itemize}
	\item \textbf{Architecture Definition}
	      \begin{itemize}
		      \item Design the client-server architecture.
		      \item Select technologies for the frontend (React, Vite, Three.js,
		            D3.js) and backend (Flask, Python).
		      \item Draft the RESTful API that will connect frontend and backend.
	      \end{itemize}
	\item \textbf{Orbital Trajectory Simulation Development}
	      \begin{itemize}
		      \item Implement orbital propagation using Keplerian parameters and
		            equations.
		      \item Perform an initial validation of the simulation with reference
		            data.
		      \item Configure the NASA NEO API to retrieve near-Earth meteor data.
	      \end{itemize}
	\item \textbf{Initial Frontend Design}
	      \begin{itemize}
		      \item Create a basic interface with React and Vite.
		      \item Implement the foundational structure for 3D trajectory
		            visualization using Three.js.
	      \end{itemize}
\end{enumerate}

\subsubsection*{Day 1 Deliverables}
\begin{itemize}
	\item Basic functional prototype with the initial user interface and an early
	      3D visualization of the meteor trajectory.
	\item Preliminary technical documentation describing the system design and
	      architecture.
	\item Initial orbital trajectory simulation results.
	\item Initial backend code covering the trajectory simulation and integration
	      with the NASA NEO API.
\end{itemize}


\subsection{Day 2: Complete Development, Validation, and Documentation}
\subsubsection*{Day 2 Objectives}
\begin{itemize}
	\item Complete development of the impact effects (crater, tsunamis,
	      earthquakes).
	\item Integrate the NASA and USGS APIs for meteor and seismic data.
	\item Implement real-time interactivity.
	\item Execute testing and optimize performance.
\end{itemize}

\subsubsection*{Activities}
\begin{enumerate}
	\item \textbf{Impact Effects Development}
	      \begin{itemize}
		      \item Implement calculations for crater formation and meteor kinetic
		            energy.
		      \item Simulate secondary impact effects (tsunamis and seismic activity)
		            using USGS data.
	      \end{itemize}
	\item \textbf{Integration of NASA and USGS Data}
	      \begin{itemize}
		      \item Integrate the NASA NEO API to fetch near-Earth meteor data in
		            real time.
		      \item Integrate the USGS Earthquake Catalog API to model the seismic and
		            geological effects of the impact.
	      \end{itemize}
	\item \textbf{Interactivity and Visualization Development}
	      \begin{itemize}
		      \item Build interactive sliders so users can adjust meteor parameters
		            (size, velocity, etc.).
		      \item Implement 3D visualization of the meteor trajectory and impact
		            effects.
		      \item Update 2D maps in real time to display tsunami- and
		            earthquake-affected areas.
	      \end{itemize}
	\item \textbf{Testing and Optimization}
	      \begin{itemize}
		      \item Conduct usability tests to ensure the tool is easy to use and
		            understand.
		      \item Optimize performance in both the frontend (3D visualization) and
		            the simulation calculations.
	      \end{itemize}
\end{enumerate}

\subsubsection*{Day 2 Deliverables}
\begin{itemize}
	\item Complete simulation of the meteor trajectory and impact effects
	      (crater formation, tsunamis, earthquakes).
	\item Interactive interface with sliders to modify parameters and view
	      real-time results.
	\item Final prototype of the web tool with 3D visualization and interactive
	      2D maps.
	\item Usability testing outcomes and performance optimizations.
	\item Final project documentation, including usage instructions and technical
	      details.
\end{itemize}
% --- REFERENCIAS BIBLIOGRAFICAS ---
\begin{thebibliography}{00}

	\bibitem{collins2005}
	G. S. Collins, H. J. Melosh, and R. A. Marcus, ``Earth impact effects program: A web-based computer program for calculating the regional environmental consequences of a meteoroid impact on earth,'' \textit{Meteoritics \& Planetary Science}, vol. 40, no. 6, pp. 817--840, 2005. doi:10.1111/j.1945-5100.2005.tb00157.x

	\bibitem{herrick2006}
	R. R. Herrick, ``Updates regarding the resurfacing of venusian impact craters,'' in \textit{Lunar and Planet. Sci. Conf. XXXVII}, p. Abs. 1588, Lunar and Planetary Institute, Houston, Texas, 2006.

	\bibitem{herrick1997}
	R. R. Herrick, V. L. Sharpton, M. C. Malin, S. N. Lyons, and K. Feely, ``Morphology and Morphometry of Impact Craters,'' in \textit{Venus II: Geology, Geophysics, Atmosphere, and Solar Wind Environment}, S. W. Bougher, D. M. Hunten, and R. J. Phillips, Eds., p. 1015, 1997.

	\bibitem{holsapple1993}
	K. A. Holsapple, ``The scaling of impact processes in planetary sciences,'' \textit{Ann. Rev. Earth Planet. Sci.}, vol. 21, pp. 333--373, 1993.

	\bibitem{mckinnon1985}
	W. B. McKinnon and P. M. Schenk, ``Ejecta blanket scaling on the Moon and Mercury - inferences for projectile populations,'' in \textit{Lunar and Planet. Sci. Conf. Proceedings XVI}, pp. 544--545, Lunar and Planetary Institute, Houston, Texas, 1985.

	\bibitem{wuennemann2010}
	K. Wünnemann, G. S. Collins, and R. Weiss, ``Impact of a cosmic body into earth’s ocean and the generation of large tsunami waves: Insight from numerical modeling,'' \textit{Reviews of Geophysics}, vol. 48, no. 4, 2010. doi:10.1029/2009RG000308

	\bibitem{melosh1989}
	H. J. Melosh, \textit{Impact Cratering: A Geologic Process}. Oxford University Press, 1989.

	\bibitem{housen2011}
	K. R. Housen and K. A. Holsapple, ``Ejecta from impact craters,'' \textit{Icarus}, vol. 211, pp. 856--875, 2011. doi:10.1016/j.icarus.2010.09.017

	\bibitem{mcgetchin1973}
	T. R. McGetchin, M. Settle, and J. W. Head, ``Lunar impact ejection and crater growth,'' \textit{Journal of Geophysical Research}, vol. 78, no. 11, pp. 10847--10863, 1973. doi:10.1029/JB078i023p10847

	\bibitem{chyba1993}
	C. F. Chyba, P. J. Thomas, and K. J. Zahnle, ``The 1908 Tunguska explosion: Atmospheric disruption of a stony meteor,'' \textit{Nature}, vol. 361, pp. 40--44, 1993. doi:10.1038/361040a0

	\bibitem{popova2013}
	O. P. Popova et al., ``Chelyabinsk Airburst, Damage Assessment, Meteorite Recovery, and Characterization,'' \textit{Science}, vol. 342, no. 6162, pp. 1069--1073, 2013. doi:10.1126/science.1242642

	\bibitem{kingery1984}
	C. N. Kingery and G. Bulmash, ``Airblast Parameters from TNT Spherical Air Burst and Hemispherical Surface Burst,'' Technical Report ARBRL-TR-02555, U.S. Army Ballistic Research Laboratory, Aberdeen Proving Ground, 1984.

	\bibitem{ufc334002}
	UFC 3-340-02, ``Structures to Resist the Effects of Accidental Explosions,'' U.S. Department of Defense, 2008 (Change 2, 2014).

	\bibitem{glasstone1977}
	S. Glasstone and P. J. Dolan, \textit{The Effects of Nuclear Weapons}. U.S. Department of Defense and U.S. Department of Energy, 1977.

	\bibitem{ward2000}
	S. N. Ward and E. Asphaug, ``Meteor impact tsunami: A probabilistic hazard assessment,'' \textit{Icarus}, vol. 145, pp. 64--78, 2000. doi:10.1006/icar.1999.6336

	\bibitem{nasaEllipticalOrbit}
	NASA Goddard Space Flight Center, ``Elliptical Orbit Design,'' Mission Visualization \url{https://nasa.github.io/mission-viz/RMarkdown/Elliptical_Orbit_Design.html}, accessed Oct. 2024.

\end{thebibliography}
\end{document}
